
\documentclass{article} % For LaTeX2e
\usepackage{iclr2016_conference,times}
\usepackage[pdftex]{graphicx}
\DeclareGraphicsExtensions{.pdf,.png}
\usepackage{url}
\usepackage{ragged2e}
\usepackage{times}
%\usepackage{epsfig}
\usepackage{array}
\usepackage{amsmath}
\usepackage{amssymb}
\usepackage{caption}
\usepackage{subcaption}
\usepackage[inline]{enumitem}
%\usepackage{xcolor}
\newcommand{\source}{{\boldsymbol{\mathbf{s}}}}\newcommand{\guide}{\boldsymbol{\mathbf{g}}}\newcommand{\adv}{\boldsymbol{\mathbf{\alpha}}}\newcommand{\neigh}[2]{\ensuremath{n_{#1} ({#2})}}\newcommand{\Neighs}[2]{\ensuremath{\mathcal{N}}_{{#1}} ({#2})}\newcommand{\NNavg}[2]{\ensuremath{a_{#1} ({#2})}}\newcommand{\NNz}[2]{\ensuremath{z_{#1} ({#2})}}\newcommand{\class}[1]{C ({#1})}\newcommand{\dist}[2]{\ensuremath{D ({#1}, {#2})}}\newcommand{\rank}[2]{\ensuremath{r_{#1} ({#2})}}\newcommand{\rankdiff}[1]{\ensuremath{\Delta{r_{#1}}}}\newcommand{\dlike}[2]{\ensuremath{\Delta L({#1}, {#2})}}%Difference in likelihood\newcommand{\AC}[2]{\ensuremath{\Delta L({#1}, {#2})}}%Angular consistency

\newcommand{\T}{\ensuremath{\top}}
\captionsetup[subfigure]{subrefformat=simple,labelformat=simple}
\renewcommand\thesubfigure{(\alph{subfigure})}

\definecolor{orange}{rgb}{1,0.5,0}
\newcommand{\yanshuai}[1]{\textcolor{red}{\textbf{[Y: #1]}}}
\newcommand{\david}[1]{\textcolor{blue}{\textbf{[D: #1]}}}
\newcommand{\sara}[1]{\textcolor{magenta}{\textbf{[S: #1]}}}
\newcommand{\fartash}[1]{\textcolor{cyan}{\textbf{[F: #1]}}}
\newcommand{\ydel}[1]{\textcolor{red}{\textbf{[Y Delete: #1]}}}
\newcommand{\yadd}[1]{\textcolor{orange}{\textbf{[Y Add: #1]}}}

%% \newcommand{\yanshuai}[1]{\textcolor{red}{\textbf{}}}
%% \newcommand{\david}[1]{\textcolor{blue}{\textbf{}}}
%% \newcommand{\sara}[1]{\textcolor{blue}{\textbf{}}}
%% \newcommand{\fartash}[1]{\textcolor{cyan}{\textbf{}}}

\newcommand{\comment}[1]{}

 %% \newcommand{\yanshuai}[1]{\textcolor{red}{\textbf{}}}
 %% \newcommand{\david}[1]{\textcolor{blue}{\textbf{}}}
 %% \newcommand{\sara}[1]{\textcolor{blue}{\textbf{}}}
 %% \newcommand{\fartash}[1]{\textcolor{cyan}{\textbf{}}}
%%
\newcommand{\beginsupplement}{%
\renewcommand{\thesection}{}% Remove section references...
\renewcommand{\thesubsection}{S\arabic{subsection}}%... from subsections
        \setcounter{table}{0}
        \renewcommand{\thetable}{S\arabic{table}}%
        \setcounter{figure}{0}
        \renewcommand{\thefigure}{S\arabic{figure}}%
     }
\usepackage[pagebackref=true,breaklinks=true,letterpaper=true,colorlinks,bookmarks=false]{hyperref}

%%%%%%%%%%%%%%%%%%%%%%
\usepackage{onimage}
\tikzset{
     <new style name>/.style={
        fill=black,
        text=white,
        font=\fontfamily{phv}\selectfont\Small\bfseries
    }
}
\tikzset{
    image label/.style={
        every node/.style={
            fill=black,
            text=white,
            font=\fontfamily{phv}\selectfont\tiny\bfseries,
            anchor=south east,
            xshift=0.5cm,
            yshift=-0.5cm,
            at={(0,1)}
        }
    }
}
%%%%%%%%%%%%%%%%%%%%%%

%%\author{Antiquus S.~Hippocampus, Natalia Cerebro \& Amelie P. Amygdale \thanks{ Use footnote for providing further information
%% about author (webpage, alternative address)---\emph{not} for acknowledging
%% funding agencies.  Funding acknowledgements go at the end of the paper.} \\
%% Department of Computer Science\\
%% Cranberry-Lemon University\\
%% Pittsburgh, PA 15213, USA \\
%% \texttt{\{hippo,brain,jen\}@cs.cranberry-lemon.edu} \\
%%\And

\title{Adversarial Manipulation of \\Deep Representations}
\author{
    Sara Sabour $^{* 1}$, Yanshuai Cao\thanks{The first two authors contributed
    equally.}~~$^{1,2}$, Fartash Faghri$^{1,2}$ \& David J.  Fleet$^1$\\
    $^1$ Department of Computer Science,
    University of Toronto, Canada\\
    $^2$ Architech Labs, Toronto, Canada\\
    \texttt{\{saaraa,caoy,faghri,fleet\}@cs.toronto.edu}    }

% The \author macro works with any number of authors. There are two commands
% used to separate the names and addresses of multiple authors: \And and \AND.
%
% Using \And between authors leaves it to \LaTeX{} to determine where to break
% the lines. Using \AND forces a linebreak at that point. So, if \LaTeX{}
% puts 3 of 4 authors names on the first line, and the last on the second
% line, try using \AND instead of \And before the third author name.

\newcommand{\fix}{\marginpar{FIX}}
\newcommand{\new}{\marginpar{NEW}}

\iclrfinalcopy % Uncomment for camera-ready version

\begin{document}


\maketitle

\begin{abstract}
\vspace*{-0.1cm}
We show that the image representations in a deep neural network
(DNN) can be manipulated to mimic those of other natural images,
with only minor, imperceptible perturbations to the original image.
Previous methods for generating adversarial images focused on image
perturbations designed to produce erroneous class labels.  Here we
instead concentrate on the internal layers of DNN representations,
to produce a new class of adversarial images that differs qualitatively
from others. While the adversary is perceptually similar to one image,
its internal representation appears remarkably similar to a different
image, from a different class and bearing little if any apparent
similarity to the input.  Further, they appear generic and consistent
with the space of natural images.  This phenomenon demonstrates the
possibility to trick a DNN to confound almost any image with any other
chosen image, and raises questions about DNN representations, as well
as the properties of natural images themselves.

\vspace*{-0.1cm}
\end{abstract}

\section{Introduction}\vspace*{-0.1cm}%% \vspace*{-0.2cm}%% \begin{figure}[h]%% \centering%% \includegraphics[width=.15\linewidth]{./imgs/technical-illustration-v3.png}%% \caption{Summary} \label{fig:illustrate}%% \vspace*{-0.2cm}%% \end{figure}%% \vspace*{-0.2cm}

Recent papers have shown that deep neural networks (DNNs) for image
classification can be fooled, often using relatively simple methods to
generate so-called {\em adversarial images}\citep{FawziEtalICLR2015, GoodfellowEtalICLR2015, GuRigazioNIPSWorkshop2014,
NguyenEtAlCVPR2015, SzegedyElatICLR2014, Tabacof2015exploring}.
% One such category of {\em adversarial images} is designed to disrupt image % classification, even though such adversaries differ almost imperceptibly % from the original source images
The existence of adversarial images is important, not
just because they reveal weaknesses in learned representations
and classifiers, but because 1) they provide opportunities to explore
fundamental questions about the nature of DNNs, e.g., whether they are
inherent in the network structure per se or in the learned models, and
2) such adversarial images might be harnessed to improve learning
algorithms that yield better generalization and robustness
\citep{GoodfellowEtalICLR2015, GuRigazioNIPSWorkshop2014}.

Research on adversarial images to date has focused mainly on disrupting
classification, i.e., on algorithms that produce images classified with
labels that are patently inconsistent with human perception.
Given the large, potentially unbounded regions of feature space associated
with a given class label, it may not be surprising that it is easy to
disrupt classification.
In this paper, in constrast to such {\em label adversaries}, we consider
a new, somewhat more incidious class of adversarial images, called
{\em feature adversaries}, which are confused with other images not
just in the class label, but in their internal representations as well.

Given a source image, a target (guide) image, and a trained DNN, we
find small perturbations to the source image that produce an internal
representation that is remarkably similar to that of the guide image,
and hence far from that of the source.
With this new class of adversarial phenomena  we demonstrate that it
is possible to fool a DNN to confound almost any image with any other
chosen image.
We further show that the deep representations of such adversarial images
are not outliers per se.  Rather, they appear generic, indistinguishable
from representations of natural images at multiple layers of a DNN.
This phenomena raises questions about DNN representations, as well
as the properties of natural images themselves.

\comment{This phenomenon raises questions about DNN
representations, as well as the properties of natural images themselves.}\comment{{{
Research on adversarial images to date has focused mainly on disrupting
classification, i.e., producing images classified with labels that are patently
inconsistent with human perception. This phenomenon is not entirely
interesting, because the disrupted image representation in the layer before
classification only needs to lie within a large and potentially unbounded
region of the feature space corresponding to the designated erroneous class,
without requiring the disrupted representation to be like real data,
i.e., representation of some real natural image.
Coupled with the fact
that input space usually has much more degrees of freedom than the final
representation space (for e.g.\ , $\sim10^{5}$D vs $\sim10^{3}D$ for
typical ImageNet classification nets), it is not surprising that labels
could be disrupted easily. Indeed, it has been shown that lack of
adversarial robustness is not limited to DNNs\citep{GoodfellowEtalICLR2015},
and is a problem for classifiers in
general \citep{FawziEtalICLR2015,fawzi2015analysis}. Orthogonal to the
ongoing development of theory on adversarial perturbation for classification,
in this paper we introduce a more disturbing category of adversarial
images that are confused with other images, not just in the class label,
but in the internal representation. In short, this new adversarial
perturbation phenomenon demonstrates that it is possible to trick a
DNN to confound almost any image with any other chosen image.
}}}%% focus directly on properties of representations learned by %% deep networks and we introduce%% What could be more interesting is adversaries %% that their internal representations at much higher dimensions and even over %% complete projections fools the model, i.e., while the input image changes %% almost imperceptively the internal representation resembles of another %% natural image.\comment{{{
Deep neural networks (DNN) have shown pervasive successes in compute vision, from classification\cite{}, detection\cite{}, segmentation\cite{} to action recognition\cite{}, etc. However, some recent discoveries showed that advarsarially designed pixel value perturbation can cause DNNs to misclassify\cite{}; while another work demonstrated that on completely unrecognizable and unnatural images, DNNs can produce high confidence predictions\cite{}. These images specifically designed to trick a given model are called {\em adversarial examples}. There has been conflicting views about the nature and relevance, especially when they are shown to exist for other classifiers as well, rather than specific to DNNs\cite{}. Regardless the truth about adversarial phenomenon on classification labels, in this work, we show the existence of a worse type of adversarial perturbation that can cause DNNs to confuse any natural image with any other given image according to the internal representation, rather than classification output. Because the learned representation is central to modern computer vision systems that rely on DNNs, our work shows that despite the widespread euphoria about recent successes, there is something troubling about the DNN representation of natural images.

%% Conflicting explanations and evidences have been
%%  about showed that there are small pertuabation to inputs or. Central to the success of DNNs is their ability to learn hierarchical distributed representation from massive amount of data.
%% In this work,
Specifically, given a source image, a target image, and the DNN, our proposed optimization finds small perturbations to the source image, which can cause the internal representation on a given layer and above to be very similar to that of the target image, and dissimilar to the representation of the source (Fig.\ref{fig:summary} \yanshuai{TBD: make a illustration like the ones we drawn on board about image space and feature space}). Furthermore, we show that the representation of the adversarial example is an inlier with respect to nearby points in the feature space, making it indistinguishable in terms of representation from other natural images in the vicinity of the target, while visually similar to the source for humans. We also demonstrate that the fact about being inlier does not hold for previously proposed adversarial examples for forcing misclassification, making our proposed adversarial examples qualitatively different. Finally, we demonstrate that previous explanations for adversarial examples do not account for our observations.

% Given the widespread use of
%\yanshuai{To be continued}

%% , which given any pair of source and target natural images, can create small and often imperceptible changes to the source image, but which make the internal representation to be very similar to
%% For instance, when trained on a large generic natural image datasets such as the ImageNet\cite{}, the representation learned by convolutional neural networks (convnets) has been shown to generalize well to many other tasks and datasets\cite{}. Such pre-trained or fine-tuned deep feature extractors have become the cornerstone in many modern computer vision applications\cite{}. However, despite the euphoria, recent works showed that DNNs can be tricked into producing erroneous prediction labels in bizarre ways when facing adversarial images\cite{}. While such adversarial examples against classification outputs are interesting, it is unclear whether the
}}}
\section{Related Work}\label{related}\vspace*{-0.1cm}% There are two main approaches to construct adversarial images % that disrupt DNN classification.  These approaches differ % fundamentally in whether they reveal the behavior of DNNs in % the neighborhood of natural images, or the adversarial images % themselves are overtly unnatural.

Several methods for generating adversarial images have appeared in
recent years.  \cite{NguyenEtAlCVPR2015} describe an evolutionary
algorithm to generate images comprising 2D patterns that are classified
by DNNs as common objects with high confidence (often $99\%$). While
interesting, such adversarial images are quite different from the
natural images used as training data.  Because natural images only
occupy a small volume of the space of all possible images, it is
not surprising that discriminative DNNs trained on natural
images have trouble coping with such out-of-sample data.

\cite{SzegedyElatICLR2014} focused on adversarial images that appear
natural. They used gradient-based optimization on the classification
loss, with respect to the image perturbation, $\epsilon$.  The
magnitude of the perturbation is penalized ensure that the perturbation
is not perceptually salient.  Given an image $I$, a DNN classifier $f$,
and an erroneous label $\ell $, they find the perturbation $\epsilon$ that
minimizes ${loss(f(I+\epsilon), \ell) + c\|\epsilon\|^2}$.
% \begin{equation}% \phantom{E=}loss(f(I+\epsilon), l) + c\|\epsilon\|^2  ~.% \end{equation}
Here, $c$ is chosen by line-search to find the smallest $\epsilon$ that
achieves $f(I+\epsilon) = \ell$. The authors argue that the resulting
adversarial images occupy low probability ``pockets'' in the manifold,
acting like ``blind spots'' to the DNN. The adversarial construction in
our paper extends the approach of \cite{SzegedyElatICLR2014}. In
Sec.~\ref{method}, we use gradient-based optimization to find small image
perturbations.  But instead of inducing misclassification, we induce dramatic
changes in the internal DNN representation.

Later work by \cite{GoodfellowEtalICLR2015} showed that adversarial images
are more common, and can be found by taking steps in the direction of the
gradient of $loss(f(I+\epsilon), \ell)$.  \cite{GoodfellowEtalICLR2015}
also show that adversarial examples exist for other models, including
linear classifiers.  They argue that the problem arises when models
are ``too linear". \cite{FawziEtalICLR2015} later propose a more general
framework to explain adversarial images,
formalizing the intuition that the problem occurs when DNNs and other
models are not sufficiently ``flexible'' for the given classification task.

In Sec.~\ref{sec:experiments}, we show that our new category of adversarial
images exhibits qualitatively different properties from those above.
In particular, the DNN representations of our adversarial images are
very similar to those of natural images.  They do not appear unnatural
in any obvious way, except for the fact that they remain inconsistent
with human perception.

%% \fartash{This is now only a short list of related work that has cited the initial work of 'Intriguing...'}%% Main related works:%% Initial work was done in \cite{szegedy2013intriguing} suggesting that given an image, for all possible choices of labels, an adversarial example can be generated that the network condfidently classifies the example as the target class.%% \cite{nguyen2014deep} Fooling convnets with examples generated by an evolutionary algorithm%% \cite{goodfellow2014explaining} Linear Perturbation theory%% \cite{mahendran2014understanding} Inverting convnets%% \cite{goodfellow2014generative} Generative Adversarial Nets%% \cite{gu2014towards} Works on solving the problem%% \cite{fawzifundamental} \cite{fawzi2015analysis} Works on theory%% \cite{guo2015cnn} We are undermining works like this%% \cite{wei2015understanding} \cite{tabacof2015exploring} Related work regarding internal representation
\section{Adversarial Image Generation}\label{method}

Let $I_s$ and $I_g$ denote the {\em source} and {\em guide} images.
Let $\phi_k$ be the mapping from an image to its internal DNN representation
at layer $k$.  Our goal is to find a new image, $I_\alpha$, such that
the Euclidian distance between $\phi_k(I_\alpha)$ and $\phi_k(I_g)$ is
as small as possible, while $I_\alpha$ remains close to the source $I_s$.
More precisely, $I_\alpha$ is defined to be the solution to a
constrained optimization problem:
\begin{align}
I_{\alpha} = \arg\min_{I} \,  \| \, \phi_k(I)-\phi_k(I_g) \,\|^{2}_2 &
\phantom{=E}
\label{adv_objective} \\
\text{subject to}~~ \| I - I_s \|_{\infty} < \delta &\phantom{=E}
\label{infnorm_bound}
\end{align}
The constraint on the distance between $I_\alpha$ and $I_s$ is formulated
in terms of the $L_{\infty}$ norm to limit the maximum deviation of any
single pixel color to $\delta$.  The goal is to constrain the degree
to which the perturbation is perceptible.  While the $L_\infty$
norm is not the best available measure of human visual discriminability
(e.g., compared to SSIM \citep{WangEtalPAMI2004}), it is superior to
the $L_2$ norm often used by others.

Rather than optimizing $\delta$ for each image,
% \comment{ as done to find the weight on the perturbation penalty % in \cite{SzegedyElatICLR2014},}
we find that a fixed value of $\delta = 10$ (out of 255) produces
compelling adversarial images with negligible perceptual distortion.
Further, it works well with different intermediate layers, different
networks and most images.
% This simplifies adversarial generation and analysis.
We only set $\delta$ larger when optimizing lower layers,
close to the input (e.g., see Fig.\\ref{fig:delta_layer}).  As $\delta$
increases distortion becomes perceptible, but there is little or no perceptible
trace of the guide image in the distortion.
For numerical optimization,  we use l-BFGS-b, with the inequality
(\ref{infnorm_bound}) expressed as a box constraint around $I_s$.

%% For notational convenience below we use $\source$, $\guide$, and $\adv$ %% to denote source, guide and adversarial images.  We also use $\adv_{ij}^k$ %% to denote the DNN representation at layer $k$, from the adversarial image %% built from source $i$ and guide $j$.\makeatletter\newcommand{\thickhline}{%
    \noalign {\ifnum 0=`}\fi \hrule height 1pt
    \futurelet \reserved@a \@xhline
}\newcolumntype{"}{@{\hskip\tabcolsep\vrule width 4pt\hskip\tabcolsep}}\makeatother\begin{figure*}[t]
\centering
\renewcommand{\arraystretch}{1}
\setlength\tabcolsep{2pt}
\begin{tabular}{ | >{\centering\arraybackslash} m{\dimexpr 0.115\linewidth}
>{\centering\arraybackslash}m{\dimexpr 0.115\linewidth} |
>{\centering\arraybackslash}m{\dimexpr 0.115\linewidth}
>{\centering\arraybackslash}m{\dimexpr 0.115\linewidth} |
>{\centering\arraybackslash}m{\dimexpr 0.115\linewidth}
>{\centering\arraybackslash}m{\dimexpr 0.115\linewidth} |
>{\centering\arraybackslash}m{\dimexpr 0.115\linewidth}
>{\centering\arraybackslash}m{\dimexpr 0.115\linewidth}@{} | }
\hline{\footnotesize Source} &
{\footnotesize Guide} &
{\footnotesize  $I_{\adv}^{\text{FC}7}\!,\delta\!=\!5$} &
%{\footnotesize $\Delta_\source(\adv^{\text{FC}7})_5 $} &
{\footnotesize $\Delta_\source $} &
{\footnotesize $I_{\adv}^{P5}\!,\delta\!=\!10$} &
%{\footnotesize $\Delta_\source(\adv^{P5})_{10}$} &
{\footnotesize $\Delta_\source$} &
{\footnotesize $I_{\adv}^{C3}\!,\delta\!=\!{15}$} &
%{\footnotesize $\Delta_\source(\adv^{C3})_{20}$} \\
{\footnotesize $\Delta_\source$} \\
\hline
\includegraphics[width=\linewidth,height=.75\linewidth]{./imgs/pie.png} &
\includegraphics[width=\linewidth,height=.75\linewidth]{./imgs/27.png} &
\includegraphics[width=\linewidth,height=.75\linewidth]{./imgs/pie_t5_g7_fc7/27img.png} &
\includegraphics[width=\linewidth,height=.75\linewidth]{./imgs/pie_t5_g7_fc7/27dif.png} &\iffalse
\includegraphics[width=\linewidth,height=.75\linewidth]{./imgs/pie_t10_g7_fc7/27img.png} &
\includegraphics[width=\linewidth,height=.75\linewidth]{./imgs/pie_t10_g7_fc7/27dif.png} &\fi
\includegraphics[width=\linewidth,height=.75\linewidth]{./imgs/pie_t10_g7_pool5/27img.png} &
\includegraphics[width=\linewidth,height=.75\linewidth]{./imgs/pie_t10_g7_pool5/27dif.png} &
\includegraphics[width=\linewidth,height=.75\linewidth]{./imgs/pie_27_c3.png} &
\includegraphics[width=\linewidth,height=.75\linewidth]{./imgs/dif-pie_27_c3.png} \\
\includegraphics[width=\linewidth,height=.75\linewidth]{./imgs/pie.png} &
\includegraphics[width=\linewidth,height=.75\linewidth]{./imgs/927.png} &
\includegraphics[width=\linewidth,height=.75\linewidth]{./imgs/pie_t5_g7_fc7/927img.png} &
\includegraphics[width=\linewidth,height=.75\linewidth]{./imgs/pie_t5_g7_fc7/927dif.png} &\iffalse
\includegraphics[width=\linewidth,height=.75\linewidth]{./imgs/pie_t10_g7_fc7/927img.png} &
\includegraphics[width=\linewidth,height=.75\linewidth]{./imgs/pie_t10_g7_fc7/927dif.png} &\fi
\includegraphics[width=\linewidth,height=.75\linewidth]{./imgs/pie_t10_g7_pool5/927img.png} &
\includegraphics[width=\linewidth,height=.75\linewidth]{./imgs/pie_t10_g7_pool5/927dif.png} &
\includegraphics[width=\linewidth,height=.75\linewidth]{./imgs/pie_927_c3.png} &
\includegraphics[width=\linewidth,height=.75\linewidth]{./imgs/dif-pie_927_c3.png} \\
\iftrue
\includegraphics[width=\linewidth,height=.75\linewidth]{./imgs/train_378.png} &
\includegraphics[width=\linewidth,height=.75\linewidth]{./imgs/927.png} &
\includegraphics[width=\linewidth,height=.75\linewidth]{./imgs/train_378_t5_g7_fc7/927img.png} &
\includegraphics[width=\linewidth,height=.75\linewidth]{./imgs/train_378_t5_g7_fc7/927dif.png} &\iffalse
\includegraphics[width=\linewidth,height=.75\linewidth]{./imgs/train_378_t10_g7_fc7/927img.png} &
\includegraphics[width=\linewidth,height=.75\linewidth]{./imgs/train_378_t10_g7_fc7/927dif.png} &\fi
\includegraphics[width=\linewidth,height=.75\linewidth]{./imgs/train_378_t10_g7_pool5/927img.png} &
\includegraphics[width=\linewidth,height=.75\linewidth]{./imgs/train_378_t10_g7_pool5/927dif.png} &
\includegraphics[width=\linewidth,height=.75\linewidth]{./imgs/378_927_c3.png} &
\includegraphics[width=\linewidth,height=.75\linewidth]{./imgs/dif-378_927_c3.png} \\
\fi
\iffalse
\includegraphics[width=\linewidth,height=.75\linewidth]{./imgs/train_665.png} &
\includegraphics[width=\linewidth,height=.75\linewidth]{./imgs/27.png} &
\includegraphics[width=\linewidth,height=.75\linewidth]{./imgs/train_665_t5_g7_fc7/27img.png} &
\includegraphics[width=\linewidth,height=.75\linewidth]{./imgs/train_665_t5_g7_fc7/27dif.png} &
\includegraphics[width=\linewidth,height=.75\linewidth]{./imgs/train_665_t10_g7_fc7/27img.png} &
\includegraphics[width=\linewidth,height=.75\linewidth]{./imgs/train_665_t10_g7_fc7/27dif.png} &
\includegraphics[width=\linewidth,height=.75\linewidth]{./imgs/train_665_t10_g7_pool5/27img.png} &
\includegraphics[width=\linewidth,height=.75\linewidth]{./imgs/train_665_t10_g7_pool5/27dif.png} &
\includegraphics[width=\linewidth,height=.75\linewidth]{./imgs/train_665_t20_g7_conv3/27img.png} &
\includegraphics[width=\linewidth,height=.75\linewidth]{./imgs/train_665_t20_g7_conv3/27dif.png} \\
\includegraphics[width=\linewidth,height=.75\linewidth]{./imgs/train_665.png} &
\includegraphics[width=\linewidth,height=.75\linewidth]{./imgs/817.png} &
\includegraphics[width=\linewidth,height=.75\linewidth]{./imgs/train_665_t5_g7_fc7/817img.png} &
\includegraphics[width=\linewidth,height=.75\linewidth]{./imgs/train_665_t5_g7_fc7/817dif.png} &
\includegraphics[width=\linewidth,height=.75\linewidth]{./imgs/train_665_t10_g7_fc7/817img.png} &
\includegraphics[width=\linewidth,height=.75\linewidth]{./imgs/train_665_t10_g7_fc7/817dif.png} &
\includegraphics[width=\linewidth,height=.75\linewidth]{./imgs/train_665_t10_g7_pool5/817img.png} &
\includegraphics[width=\linewidth,height=.75\linewidth]{./imgs/train_665_t10_g7_pool5/817dif.png} &
\includegraphics[width=\linewidth,height=.75\linewidth]{./imgs/train_665_t20_g7_conv3/817img.png} &
\includegraphics[width=\linewidth,height=.75\linewidth]{./imgs/train_665_t20_g7_conv3/817dif.png} \\
\includegraphics[width=\linewidth,height=.75\linewidth]{./imgs/train_378.png} &
\includegraphics[width=\linewidth,height=.75\linewidth]{./imgs/437.png} &
\includegraphics[width=\linewidth,height=.75\linewidth]{./imgs/train_378_t5_g7_fc7/437img.png} &
\includegraphics[width=\linewidth,height=.75\linewidth]{./imgs/train_378_t5_g7_fc7/437dif.png} &
\includegraphics[width=\linewidth,height=.75\linewidth]{./imgs/train_378_t10_g7_fc7/437img.png} &
\includegraphics[width=\linewidth,height=.75\linewidth]{./imgs/train_378_t10_g7_fc7/437dif.png} &
\includegraphics[width=\linewidth,height=.75\linewidth]{./imgs/train_378_t10_g7_pool5/437img.png} &
\includegraphics[width=\linewidth,height=.75\linewidth]{./imgs/train_378_t10_g7_pool5/437dif.png} &
\includegraphics[width=\linewidth,height=.75\linewidth]{./imgs/train_378_t20_g7_conv3/437img.png} &
\includegraphics[width=\linewidth,height=.75\linewidth]{./imgs/train_378_t20_g7_conv3/437dif.png} \\
\includegraphics[width=\linewidth,height=.75\linewidth]{./imgs/train_378.png} &
\includegraphics[width=\linewidth,height=.75\linewidth]{./imgs/967.png} &
\includegraphics[width=\linewidth,height=.75\linewidth]{./imgs/train_378_t5_g7_fc7/967img.png} &
\includegraphics[width=\linewidth,height=.75\linewidth]{./imgs/train_378_t5_g7_fc7/967dif.png} &
\includegraphics[width=\linewidth,height=.75\linewidth]{./imgs/train_378_t10_g7_fc7/967img.png} &
\includegraphics[width=\linewidth,height=.75\linewidth]{./imgs/train_378_t10_g7_fc7/967dif.png} &
\includegraphics[width=\linewidth,height=.75\linewidth]{./imgs/train_378_t10_g7_pool5/967img.png} &
\includegraphics[width=\linewidth,height=.75\linewidth]{./imgs/train_378_t10_g7_pool5/967dif.png} &
\includegraphics[width=\linewidth,height=.75\linewidth]{./imgs/train_378_t20_g7_conv3/967img.png} &
\includegraphics[width=\linewidth,height=.75\linewidth]{./imgs/train_378_t20_g7_conv3/967dif.png} \\
\fi
\hline
\end{tabular}
\caption{Each row shows examples of adversarial images, optimized
using different layers of Caffenet (FC$7$, P$5$, and C$3$), and different
values of $\delta=(5, 10, 15)$.  Beside each adversarial image is the
difference between its corresponding source image.}
\label{fig:adv_caffenet}
\vspace*{-0.1cm}
\end{figure*}

Figure~\ref{fig:adv_caffenet} shows nine adversarial images generated
in this way, all using the well-known BVLC Caffe Reference model
(Caffenet) \citep{jia2014caffe}.
Each row in Fig.~\ref{fig:adv_caffenet} shows a source, a guide, and three
adversarial images along with their differences from the corresponding source.
The adversarial examples were optimized with different perturbation bounds
($\delta$), and using different layers, namely FC$7$ (fully connected level 7),
P$5$ (pooling layer 5), and C3 (convolution layer 3).  Inspecting the
adversarial images, one can see that larger values of $\delta$ allow more
noticeable perturbations.  That said, we have found no natural images in which
the guide image is perceptible in the adversarial image.  Nor is there
a significant amount of salient structure readily visible in the difference
images.

While the class label was not an explicit factor in the optimization, we
find that class labels assigned to adversarial images by the DNN are almost
always that of the guide.  For example, we took 100 random source-guide
pairs of images from Imagenet ILSVRC data \citep{deng2009imagenet}, and
applied optimization using layer FC7 of Caffenet, with $\delta = 10$.
We found that class labels assigned to adversarial images were never
equal to those of source images. Instead, in 95\% of cases they matched
the guide class.  This remains true for source images from training,
validation, and test ILSVRC data.

We found a similar pattern of behavior with other networks and datasets,
including AlexNet \citep{krizhevsky2012imagenet}, GoogleNet
\citep{szegedy2014going}, and VGG CNN-S \citep{chatfield2014return},
all trained on the Imagenet ILSVRC dataset.  We also used AlexNet
trained on the Places205 dataset, and on a hybrid dataset comprising 205
scene classes and 977 classes from ImageNet \citep{zhou2014learning}.
In all cases, using 100 random source-guide pairs the class labels
assigned to the adversarial images do not match the source.  Rather, in
97\% to 100\% of all cases the predicted class label is that of the guide.

% \david{I have removed most discussion of the adversarial image% representation being the 1-NN for the guide, since that may be % better placed at the beginning of section 4, to add to the results% in Fig 4 about how much closer the adversarial rep is to the guide% compared to the source.}

Like other approaches to generating adversarial images
(e.g., \cite{SzegedyElatICLR2014}), we find that those generated
on one network are usually misclassified by other networks
Using the same 100 source-guide pairs with each of the models above,
we find that, on average, 54\% of adversarial images obtained from one
network are misclassified by other networks.  That said, they are usually
not consistently classified with the same label as the guide on different
netowrks.

%%%%%%%%%%%%%%%%%%%%%%%%%%%%%%%%%%%%%%%%%%%%%%%%%%%%%%%%%%%%%%%%%%%%%%%%%%%%%%%%%%%%%%%%%%%%%%%%%%%%%%%%%%%%%%%%%%%%%%%%%%%%%%%%%%%%%%%%%%%%%%%%%%%%%%%%%%\comment{{{
\begin{table*}[ht]
\resizebox{\linewidth}{!}{\centering
\begin{tabular}{|c|c|c|c|c|c|} \hline
Model & Layer & Split sets & Out $\class{\source}$ &
In $\class{\guide}$ & $\guide=\neigh{1}{\adv}$\\
\hline
%
CaffeNet&       FC$7$&  train-train&    $100$&  $95$&   $95$\\
AlexNet&        FC$7$&  train-train&    $100$&  $97$&   $96$\\
GoogleNet&      pool5/7x7\_s1&  train-train&    $100$&  $100$&  $100$\\
VGG CNN S&      FC$7$&  train-train&    $100$&  $99$&   $99$\\
Places205 AlexNet&      FC$7$&  train-train&    $100$&  $100$&  $98$\\
Places205 Hybrid&       FC$7$&  train-train&    $100$&  $99$&   $99$\\
Flickr Style&   FC$8$ Flickr&   train-train&    $100$&  $100$&  $95$\\
Flickr Style&   FC$7$&  train-train&    $98$&   $45$&   $32$\\
\hline
CaffeNet&       FC$7$&  train-test&     $100$&  $97$&   $96$\\
AlexNet&        FC$7$&  train-test&     $100$&  $99$&   $97$\\
GoogleNet&      pool5/7x7\_s1&  train-test&     $100$&  $99$&   $99$\\
VGG CNN S&      FC$7$&  train-test&     $100$&  $99$&   $99$\\
Places205 AlexNet&      FC$7$&  train-val&      $100$&  $99$&   $99$\\
Places205 Hybrid&       FC$7$&  train-val&      $100$&  $100$&  $100$\\
Flickr Style&   FC$8$ Flickr&   train-test&     $100$&  $100$&  $95$\\
Flickr Style&   FC$7$&  train-test&     $98$&   $41$&   $31$\\
\hline
CaffeNet&       FC$7$&  val-val&        $100$&  $98$&   $0$\\
\hline
\end{tabular}}
\caption{Results for generating adversarials on various networks.
For each row, $100$ random samples are drawn from indicated splits
of the dataset and optimized for $500$ iterations.
{\it Out $\class{\source}$\/} is the number of adversarials that
are predicted by the network to have a different label from source.
{\it In $\class{\guide}$\/} of the adversarials are predicted to
be from the class of guide.  $\guide=\neigh{1}{\adv}$ means the
adversarial is so close to the guide that guide is its $1$NN\@.}
\label{tb:generalize_tb1}
\end{table*}
}}}%%%%%%%%%%\comment{{{
The Flickr Style has $80,000$ images from Flickr,
categorized into 20 classes of photo styles.

We repeat the experiments described in Section~\ref{RankAnalysis} on
other networks to show that the generated adversarial images look like
inliers in the guide class for various CNN networks. For networks trained
on Flickr style dataset we show that it is possible to find adversarials
on the last layer before classification but for layers below, sometimes
the optimization quickly ends up in local minima.

Table~\ref{tb:generalize_tb1} shows that not only our adversarials almost
always have the label of the guide, but also they are almost always so
close to the guide that the first nearest neighbor of them in the class
of the guide is the guide itself. For all networks, except for FC$7$
on Flickr, the difference in rank, $\rankdiff{3}$, is better than $-7$
which means that for half of the adversarials, we are more similar
to the guide compared to $93\%$ of other images in that class.

On the fine-tuned model on Flickr Style, we observe that we can quickly
generate adversarials that we can optimize the FC$8$, the unnormalized
class scores, to be the same as the guide with $\rankdiff{3}$ being $0$.
However, when doing the optimization on FC$7$, for $55$ examples on
train-train split and $59$ examples on train-test split, the optimization
quickly converges to local minima. The interesting observation is that,
almost always, the local minima has $\rank{3}=0$. This means that the
optimization has converged to the most dense regions of a class.






%%%%%%%%%%
\begin{table*}[ht] \resizebox{\linewidth}{!}{\centering

\begin{tabular}{|c|c|c|c|} \hline Model & Layer &Same $3$NN & Same
$2/3$ 3NN\\ \hline
%
CaffeNet&       FC$7$&  $71$&   $24$\\
AlexNet&        FC$7$&  $72$&   $25$\\
GoogleNet&      pool5/7x7\_s1&  $87$&   $13$\\
VGG CNN S&      FC$7$&  $84$&   $16$\\
Places205 AlexNet&      FC$7$&  $91$&   $9$\\
Places205 Hybrid&       FC$7$&  $85$&   $15$\\
Flickr Style&   FC$8$ Flickr&   $100$&  $0$\\
Flickr Style&   FC$7$&  $28$&   $28$\\
\hline
CaffeNet&       FC$7$&  $70$&   $29$\\
AlexNet&        FC$7$&  $79$&   $20$\\
GoogleNet&      pool5/7x7\_s1&  $97$&   $3$\\
VGG CNN S&      FC$7$&  $79$&   $20$\\
Places205 AlexNet&      FC$7$&  $91$&   $9$\\
Places205 Hybrid&       FC$7$&  $84$&   $14$\\
Flickr Style&   FC$8$ Flickr&   $100$&  $0$\\
Flickr Style&   FC$7$&  $25$&   $30$\\
\hline
CaffeNet&       FC$7$&  $78$&   $20$\\
\hline
            %
    %
    \end{tabular}}
                %
  \caption{This table shows that for most of the time the nearest neighbors of
  the guide and the adversarial are exactly the same data points. The first
  column is the number of cases where all 3 nearest neighbors are the same.
  The second column is when 2 out of 3 are the same.}
     %
 \label{tb:generalize_tb1_2}
%
\end{table*}
}}}%%%%%%%%%%%%\comment{{{
\begin{table*}[ht] \resizebox{\linewidth}{!}{\centering
        \begin{tabular}{|c|c|c|c|c|c|c|} \hline Model Name & Layer & Split sets
        & Out $\class{\source}$& In $\class{\guide}$ & $\guide=\neigh{1}{\adv}$
        & $\rankdiff{3}$ median, [min, max]\\ \hline CaffeNet&       FC$7$&
        train-train&    $100$&  $95$&   $95$&   $-5.76, [-54.83, 0.00]$\\
        AlexNet&        FC$7$&  train-train&    $100$&  $97$&   $96$&   $-5.64,
        [-38.39, 0.00]$\\ GoogleNet&      pool5/7x7\_s1&  train-train& $100$&
        $100$&  $100$&  $-1.94, [-12.87, 0.10]$\\ VGG CNN S& FC$7$&
        train-train&    $100$&  $99$&   $99$&   $-3.37, [-26.34,
    0.00]$\\ Places205 AlexNet&      FC$7$&  train-train&    $100$&  $100$&
        $98$&   $-1.25, [-18.20, 8.04]$\\ Places205 Hybrid&       FC$7$&
        train-train&    $100$&  $99$&   $99$&   $-1.30, [-8.96, 8.29]$\\ Flickr
        Style&   FC$8$ Flickr&   train-train&    $100$&  $100$&  $95$&   $0.00,
        [0.00, 0.08]$\\ Flickr Style&   FC$7$&  train-train&    $98$&   $45$&
        $32$&   $-10.28,        [-37.35, -2.39]$\\ \hline CaffeNet&
        FC$7$&  train-test&     $100$&  $97$&   $96$&   $-5.37, [-27.38,
    0.00]$\\ AlexNet&        FC$7$&  train-test&     $100$&  $99$&   $97$&
        $-6.10, [-42.98, 0.23]$\\ GoogleNet&      pool5/7x7\_s1&  train-test&
        $100$&  $99$&   $99$&   $-1.62, [-10.18, 0.51]$\\ VGG CNN S&
        FC$7$&  train-test&     $100$&  $99$&   $99$&   $-3.36, [-18.21,
    0.73]$\\ Places205 AlexNet&      FC$7$&  train-val&      $100$&  $99$&
        $99$&   $-1.16, [-8.30, 3.25]$\\ Places205 Hybrid&       FC$7$&
        train-val&      $100$&  $100$&  $100$&  $-1.38, [-7.39, 5.87]$\\ Flickr
        Style&   FC$8$ Flickr&   train-test&     $100$&  $100$&  $95$&   $0.00,
        [0.00, 0.04]$\\ Flickr Style&   FC$7$&  train-test&     $98$&   $41$&
        $31$&   $-10.94,        [-28.96, 0.00]$\\ \hline CaffeNet&       FC$7$&
        val-val&        $100$&  $98$&   $-$&    $-6.29, [-25.58, -0.08]$\\
        \hline
            %
    %
    \end{tabular}}
                %
  \caption{Results for generating adversarials on various networks.  For each
      row, $100$ random samples are drawn from indicated splits of the dataset
      and optimized for $500$ iterations. {\it Out $\class{\source}$\/} is the
      number of adversarials that are predicted by the network to have
      a different label from source.  {\it In $\class{\guide}$\/} of the
      adversarials are predicted to be from the class of guide.
  $\guide=\neigh{1}{\adv}$ means the adversarial is so close to the guide that
  guide is its $1$NN\@.  Column of $\rankdiff{3}$ shows statistics for the
  difference of ranks.}
     %
 \label{tb:generalize_tb1_prev}
%
\end{table*}



\begin{table}[]
% \resizebox{\linewidth}{!}{\centering
\begin{center}
\begin{footnotesize}
\begin{tabular}{|c|c|c|c|c|}
\hline
\multicolumn{5}{|c|}{train-train}\\
\hline
Model name      &CaffeNet       &AlexNet        &GoogleNet      &VGG CNN S \\
\hline
CaffeNet &      100,95 &        61,3 &  42,0 &  53,1 \\
AlexNet &       61,0 &  100,97 &        46,0 &  57,0 \\
GoogleNet &     33,1 &  36,0 &  100,100 &       42,0 \\
VGG CNN S &     53,0 &  55,0 &  49,1 &  100,99 \\
            \hline
            \multicolumn{5}{|c|}{train-test}\\
            \hline
Model name      &CaffeNet       &AlexNet        &GoogleNet      &VGG CNN S \\
\hline
CaffeNet &      100,97 &        67,1 &  58,0 &  65,1 \\
AlexNet &       71,1 &  100,99 &        54,0 &  67,0 \\
GoogleNet &     50,0 &  49,1 &  100,99 &        48,0 \\
VGG CNN S &     59,2 &  57,0 &  61,0 &  100,99 \\
            \hline
\end{tabular}
\end{footnotesize}
\end{center}
% }
\caption{Results for generating an adversarial for one network and testing it
on another network. The rows show the network for generating the adversarial
and the column shows the network used for testing the adversarials. For each
row, $100$ adversarials generated for Table~\ref{tb:generalize_tb1} is tested
on networks in the columns. Each cell shows the number of {\it Out
$\class{\source}$\/} and {\it In $\class{\guide}$}. } \label{tb:generalize_tb2}
%
\end{table}
}}}\comment{
The adversarial images generated by our optimization are designed to be
close, if not perceptually equivalent, to a given source image (demonstrated
in Fig.~\ref{fig:adv_caffenet}), with an internal represetation close to
the guide.
}%%%%%%%%%%%%%%%%%%%%%%%%%%%%%%%%%%%%%%%%%%%%%%%%%%%%%%%%%%%%%%%%%%%%%%%%%%%%%%%%%%%%%%%%%%%%%%%%%%%%%%%%%%%%%%%%%%%%%%%%%%%%%%%%%%%%%%%%%%%%%%%%
We next turn to consider internal representations -- do they resemble
those of the source, the guide, or some combination of the two?
One way to probe the internal representations, following
\cite{MahendranVedaldiCVPR2015}, is to invert the mapping, thereby
reconstructing images from internal representations at specific layers.
The top panel in Fig.~\ref{fig:adv_invert} shows reconstructed images
for a source-guide pair.  The {\em Input} row displays a source (left),
a guide (right) and adervarisal images optimized to match representations at
layers FC7, P5 and C3 of Caffenet (middle).  Subsequent rows show
reconstructions from the internal representations of these five
images, again from layers C3, P5 and FC7.
Note how lower layers bear more similarity to the source, while higher
layers resemble the guide.
When optimized using C3, the reconstructions
from C3 shows a mixture of source and guide.
In almost all cases we find that internal representations begin
to mimic the guide at the layer targeted by the optimization.
These reconstructions suggest that human perception and the DNN
representations of these adversarial images are clearly at odds with
one another.

The bottom panel of Fig.~\ref{fig:adv_invert} depicts FC7 and P5 activation
patterns for the source and guide images in Fig.~\ref{fig:adv_invert},
along with those for their corresponding adversarial images.
We note that the adversarial activations are sparse and much more
closely resemble the guide encoding than the source encoding.
The supplementary material includes several more examples of
adversarial images, their activation patterns, and reconstructions
from intermediate layers.

\comment{In particular, with random noise as an initial guess, they optimize
the following regularized objective using stochastic gradient descent
in order to find a reconstructed image $I_*$ with unit $L_2$ norm:
\begin{equation}
\label{eq:inv_opt}
\frac{\lVert \Phi(\sigma I_*)-\Phi(I)\rVert^2_2 }{ \lVert \Phi(I) \rVert^2_2}
    +\lambda_\alpha\mathcal{R}_\alpha(I_*)+\lambda_{V^\beta}\mathcal{R}_{V^\beta}( I_* )
\end{equation}
where the two regularizers, $\mathcal{R}_\alpha(x)$ and
$\mathcal{R}_{V^\beta}$, serve to produce natural images with a
consistent dynamic range. Some form of regularization here is clearly
necessary because the DNN mapping is many-to-one.
}\begin{figure*}[t!]
\centering
\begin{subfigure}[t]{\linewidth}{
\renewcommand{\arraystretch}{1}
\setlength\tabcolsep{2pt}
\begin{tabular}{|
>{\centering\arraybackslash}m{0.09\linewidth} |
>{\centering\arraybackslash}m{0.167\linewidth} |
>{\centering\arraybackslash}m{0.167\linewidth}
>{\centering\arraybackslash}m{0.167\linewidth}
>{\centering\arraybackslash}m{0.167\linewidth} |
>{\centering\arraybackslash}m{0.167\linewidth} | }
\hline
 & Source & $\text{FC}7$ & $\text{P}5$ & C$3$ &Guide  \\\hline
Input &
%\begin{tikzonimage}[width=\linewidth,
%height=.75\linewidth]{./imgs/train_730.png}[image label]
%\node{S}; \end{tikzonimage}
\includegraphics[width=\linewidth,height=.75\linewidth]{./imgs/train_730.png}
%& \begin{tikzonimage}[width=\linewidth, height=.75\linewidth]
%    {./imgs/train_730_t10_fc7_100043mat/orig.png}[image label]
%\node{F};\end{tikzonimage}
 &
 \includegraphics[width=\linewidth,height=.75\linewidth]{./imgs/train_730_t10_fc7_100043mat/orig.png}
 &
% \begin{tikzonimage}[width=\linewidth, height=.75\linewidth]
%    {./imgs/train_730_t10_pool5_100043mat/orig.png}[image label]
%\node{P};\end{tikzonimage}
\includegraphics[width=\linewidth,height=.75\linewidth]{./imgs/train_730_t10_pool5_100043mat/orig.png}
&
% \begin{tikzonimage}[width=\linewidth, height=.75\linewidth]
%    {./imgs/train_730_t20_conv3_100043mat/orig.png}[image label]
%\node{C};\end{tikzonimage}
\includegraphics[width=\linewidth,height=.75\linewidth]{./imgs/train_730_t20_conv3_100043mat/orig.png}
&
% \begin{tikzonimage}[width=\linewidth, height=.75\linewidth]
%    {./imgs/val_43.png}[image label] \node{G};\end{tikzonimage} \\
\includegraphics[width=\linewidth,height=.75\linewidth]{./imgs/val_43.png} \\
Inv(C$3$) &
\includegraphics[width=\linewidth,height=.75\linewidth]{./imgs/730/l09-recon.png} &
\includegraphics[width=\linewidth,height=.75\linewidth]{./imgs/train_730_t10_fc7_100043mat/l09-recon.png} &
\includegraphics[width=\linewidth,height=.75\linewidth]{./imgs/train_730_t10_pool5_100043mat/l09-recon.png} &
\includegraphics[width=\linewidth,height=.75\linewidth]{./imgs/train_730_t20_conv3_100043mat/l09-recon.png} &
\includegraphics[width=\linewidth,height=.75\linewidth]{./imgs/100043/l09-recon.png}
\\
Inv($\text{P}5$) &
\includegraphics[width=\linewidth,height=.75\linewidth]{./imgs/730/l16-recon.png} &
\includegraphics[width=\linewidth,height=.75\linewidth]{./imgs/train_730_t10_fc7_100043mat/l16-recon.png} &
\includegraphics[width=\linewidth,height=.75\linewidth]{./imgs/train_730_t10_pool5_100043mat/l16-recon.png} &
\includegraphics[width=\linewidth,height=.75\linewidth]{./imgs/train_730_t20_conv3_100043mat/l16-recon.png} &
\includegraphics[width=\linewidth,height=.75\linewidth]{./imgs/100043/l16-recon.png}
\\
Inv($\text{FC}7$) &
\includegraphics[width=\linewidth,height=.75\linewidth]{./imgs/730/l18-recon.png} &
\includegraphics[width=\linewidth,height=.75\linewidth]{./imgs/train_730_t10_fc7_100043mat/l18-recon.png} &
\includegraphics[width=\linewidth,height=.75\linewidth]{./imgs/train_730_t10_pool5_100043mat/l18-recon.png} &
\includegraphics[width=\linewidth,height=.75\linewidth]{./imgs/train_730_t20_conv3_100043mat/l18-recon.png} &
\includegraphics[width=\linewidth,height=.75\linewidth]{./imgs/100043/l18-recon.png}
\\

\hline
\end{tabular}
}
\end{subfigure}

\vspace*{0.2cm}

\begin{subfigure}[t]{\linewidth}{
\centering
\renewcommand{\arraystretch}{1}
\setlength\tabcolsep{.1pt}
\begin{tabular}{
|>{\centering\arraybackslash}m{0.205\linewidth}
>{\centering\arraybackslash}m{0.205\linewidth}
>{\centering\arraybackslash}m{0.205\linewidth}|
>{\centering\arraybackslash}m{0.125\linewidth}
>{\centering\arraybackslash}m{0.125\linewidth}
>{\centering\arraybackslash}m{0.125\linewidth}|
}
\hline
% \begin{tikzonimage}[width=\linewidth]
%     {./imgs/f7_730.png}[image label] \node{S}; \end{tikzonimage}&
\includegraphics[width=\linewidth]{./imgs/f7_730.png} &
% \begin{tikzonimage}[width=\linewidth]
%     {./imgs/f7_730_43.png}[image label] \node{F}; \end{tikzonimage}&
\includegraphics[width=\linewidth]{./imgs/f7_730_43.png} &
% \begin{tikzonimage}[width=\linewidth]
%     {./imgs/f7_43.png}[image label] \node{G}; \end{tikzonimage}&
\includegraphics[width=\linewidth]{./imgs/f7_43.png} &
% \begin{tikzonimage}[height=\linewidth, angle=90]
%     {./imgs/p5_730.png}[image label] \node{S}; \end{tikzonimage}&
\includegraphics[height=\linewidth, angle=90]{./imgs/p5_730.png} &
% \begin{tikzonimage}[height=\linewidth, angle=90]
% {./imgs/p5_730_43.png}[image label] \node{P};\end{tikzonimage}&
\includegraphics[height=\linewidth, angle=90]{./imgs/p5_730_43.png} &
% \begin{tikzonimage}[height=\linewidth, angle=90]
%    {./imgs/p5_43.png}[image label] \node{G}; \end{tikzonimage}\\
\includegraphics[height=\linewidth, angle=90]{./imgs/p5_43.png}\\
Source & FC7 Advers. & Guide & Source & P5 Advers. & Guide \\ \hline
%\multicolumn{3}{|c|}{FC$7$ Activations} & \multicolumn{3}{c|}{ P$5$
%Activations  } \\
%\hline
\end{tabular}
}
\end{subfigure}
\caption{
(Top Panel) The top row shows a source (left), a guide (right), and
three adversarial images (middle), optimized using layers FC$7$, P$5$,
and C3 of Caffenet.  The next three rows show images obtained by
inverting the DNN mapping, from layers C3, P$5$, and FC$7$
respectively \citep{MahendranVedaldiCVPR2015}.
(Lower Panel) Activation patterns are shown at layer FC7 for the source,
guide and FC7 adversarial above, and at layer P5 for the source, guide
and P5 adversarial image above.
% One can see how the internal representations begin to mimic the guide,
% beginning at the layer used in the optimization.
}
\label{fig:adv_invert}
\end{figure*}%%%%%%%%%%%%%%%%%%%%%%%%%%%%%%%%%%%%%%%%%%%%%%%%%%%%%%%%%%%%%%%%%%%%%%%%%%%%%%%%%%%%%%%%%%%%%%%%%%%%%%%%%%%%%%%%%%%%%%%%%%%%%%%%%%%%%%%%%%%%%%%%\comment{{{
\iffalse
\begin{figure*}[t!]
\centering
\renewcommand{\arraystretch}{0.3}
\setlength\tabcolsep{0pt}
\begin{tabular}{|>{\centering\arraybackslash}m{0.095\linewidth}"
>{\centering\arraybackslash}m{0.095\linewidth}>{\centering\arraybackslash}m{0.095\linewidth}
>{\centering\arraybackslash}m{0.095\linewidth}"
>{\centering\arraybackslash}m{0.095\linewidth}"
>{\centering\arraybackslash}m{0.095\linewidth}>{\centering\arraybackslash}m{0.095\linewidth}
>{\centering\arraybackslash}m{0.095\linewidth}"
>{\centering\arraybackslash}m{0.095\linewidth}|}
\hline

& & \vspace{0.2pt}$s_1$ & & \vspace{0.2pt} $g$& &\vspace{0.2pt} $s_2$ &  &\\[2ex] \hline
& \vspace{0.2pt} $\alpha^{F7}_{11}$ &\vspace{0.2pt} $\alpha^{\text{P}5}_{11}$ &  \vspace{0.2pt}$\alpha^{C3}_{11}$ &\vspace{3pt} $g_1$  &\vspace{0.2pt} $\alpha^{C3}_{21}$ & \vspace{0.2pt}$\alpha^{\text{P}5}_{21}$  & \vspace{0.2pt}$\alpha^{F7}_{21}$ & \vspace{5pt} $s_2$\\[1ex] \hline
Input &
\includegraphics[width=\linewidth,height=.75\linewidth]{./imgs/pie_t10_fc7_27mat/orig.png} &
\includegraphics[width=\linewidth,height=.75\linewidth]{./imgs/pie_t10_pool5_27mat/orig.png} &
\includegraphics[width=\linewidth,height=.75\linewidth]{./imgs/pie_t20_conv3_27mat/orig.png} &
\includegraphics[width=\linewidth,height=.75\linewidth]{./imgs/27.png} &
\includegraphics[width=\linewidth,height=.75\linewidth]{./imgs/train_730_t20_conv3_27mat/orig.png} &
\includegraphics[width=\linewidth,height=.75\linewidth]{./imgs/train_730_t10_pool5_27mat/orig.png} &
\includegraphics[width=\linewidth,height=.75\linewidth]{./imgs/train_730_t10_fc7_27mat/orig.png}
&
\includegraphics[width=\linewidth,height=.75\linewidth]{./imgs/train_730.png}
 \\
\hline
Inv($C3$) &
 \includegraphics[width=\linewidth,height=.75\linewidth]{./imgs/pie_t10_fc7_27mat/l09-recon.png}&
\includegraphics[width=\linewidth,height=.75\linewidth]{./imgs/pie_t10_pool5_27mat/l09-recon.png}&
\includegraphics[width=\linewidth,height=.75\linewidth]{./imgs/pie_t20_conv3_27mat/l09-recon.png}&
\includegraphics[width=\linewidth,height=.75\linewidth]{./imgs/27/l09-recon.png} &
\includegraphics[width=\linewidth,height=.75\linewidth]{./imgs/train_730_t20_conv3_27mat/l09-recon.png}&
\includegraphics[width=\linewidth,height=.75\linewidth]{./imgs/train_730_t10_pool5_27mat/l09-recon.png}&
\includegraphics[width=\linewidth,height=.75\linewidth]{./imgs/train_730_t10_fc7_27mat/l09-recon.png}
&
\includegraphics[width=\linewidth,height=.75\linewidth]{./imgs/730/l09-recon.png}
\\
Inv($\text{P}5$) &
\includegraphics[width=\linewidth,height=.75\linewidth]{./imgs/pie_t10_fc7_27mat/l15-recon.png}&
\includegraphics[width=\linewidth,height=.75\linewidth]{./imgs/pie_t10_pool5_27mat/l15-recon.png}&
\includegraphics[width=\linewidth,height=.75\linewidth]{./imgs/pie_t20_conv3_27mat/l15-recon.png}&
\includegraphics[width=\linewidth,height=.75\linewidth]{./imgs/27/l15-recon.png} &
\includegraphics[width=\linewidth,height=.75\linewidth]{./imgs/train_730_t20_conv3_27mat/l15-recon.png}&
\includegraphics[width=\linewidth,height=.75\linewidth]{./imgs/train_730_t10_pool5_27mat/l15-recon.png}&
\includegraphics[width=\linewidth,height=.75\linewidth]{./imgs/train_730_t10_fc7_27mat/l15-recon.png}
&
\includegraphics[width=\linewidth,height=.75\linewidth]{./imgs/730/l15-recon.png}
\\
Inv($FC6$) &
\includegraphics[width=\linewidth,height=.75\linewidth]{./imgs/pie_t10_fc7_27mat/l16-recon.png}&
\includegraphics[width=\linewidth,height=.75\linewidth]{./imgs/pie_t10_pool5_27mat/l16-recon.png}&
\includegraphics[width=\linewidth,height=.75\linewidth]{./imgs/pie_t20_conv3_27mat/l16-recon.png}&
\includegraphics[width=\linewidth,height=.75\linewidth]{./imgs/27/l16-recon.png} &
\includegraphics[width=\linewidth,height=.75\linewidth]{./imgs/train_730_t20_conv3_27mat/l16-recon.png}&
\includegraphics[width=\linewidth,height=.75\linewidth]{./imgs/train_730_t10_pool5_27mat/l16-recon.png}&
\includegraphics[width=\linewidth,height=.75\linewidth]{./imgs/train_730_t10_fc7_27mat/l16-recon.png}
&
\includegraphics[width=\linewidth,height=.75\linewidth]{./imgs/730/l16-recon.png}
\\
Inv($\text{FC}7$) &
\includegraphics[width=\linewidth,height=.75\linewidth]{./imgs/pie_t10_fc7_27mat/l18-recon.png}&
\includegraphics[width=\linewidth,height=.75\linewidth]{./imgs/pie_t10_pool5_27mat/l18-recon.png}&
\includegraphics[width=\linewidth,height=.75\linewidth]{./imgs/pie_t20_conv3_27mat/l18-recon.png}&
\includegraphics[width=\linewidth,height=.75\linewidth]{./imgs/27/l18-recon.png} &
\includegraphics[width=\linewidth,height=.75\linewidth]{./imgs/train_730_t20_conv3_27mat/l18-recon.png}&
\includegraphics[width=\linewidth,height=.75\linewidth]{./imgs/train_730_t10_pool5_27mat/l18-recon.png}&
\includegraphics[width=\linewidth,height=.75\linewidth]{./imgs/train_730_t10_fc7_27mat/l18-recon.png}
&
\includegraphics[width=\linewidth,height=.75\linewidth]{./imgs/730/l18-recon.png}
\\
\hline
\vspace{5pt}$s_1$ & \vspace{0.2pt} $\alpha^{F7}_{12}$ &\vspace{0.2pt} $\alpha^{\text{P}5}_{12}$ &  \vspace{0.2pt}$\alpha^{C3}_{12}$ &\vspace{3pt} $g_2$  &\vspace{0.2pt} $\alpha^{C3}_{22}$ & \vspace{0.2pt}$\alpha^{\text{P}5}_{22}$  & \vspace{0.2pt}$\alpha^{F7}_{22}$ &\\[3ex]
\hline
\includegraphics[width=\linewidth,height=.75\linewidth]{./imgs/pie}
&
\includegraphics[width=\linewidth,height=.75\linewidth]{./imgs/pie_t10_fc7_100043mat/orig} &
\includegraphics[width=\linewidth,height=.75\linewidth]{./imgs/pie_t10_pool5_100043mat/orig} &
\includegraphics[width=\linewidth,height=.75\linewidth]{./imgs/pie_t20_conv3_100043mat/orig} &
\includegraphics[width=\linewidth,height=.75\linewidth]{./imgs/val_43.png} &
\includegraphics[width=\linewidth,height=.75\linewidth]{./imgs/train_730_t20_conv3_100043mat/orig} &
\includegraphics[width=\linewidth,height=.75\linewidth]{./imgs/train_730_t10_pool5_100043mat/orig} &
\includegraphics[width=\linewidth,height=.75\linewidth]{./imgs/train_730_t10_fc7_100043mat/orig.png}
& Input
\\
\hline
\includegraphics[width=\linewidth,height=.75\linewidth]{./imgs/pie/l09-recon.png}
&
\includegraphics[width=\linewidth,height=.75\linewidth]{./imgs/pie_t10_fc7_100043mat/l09-recon.png}&
\includegraphics[width=\linewidth,height=.75\linewidth]{./imgs/pie_t10_pool5_100043mat/l09-recon.png}&
\includegraphics[width=\linewidth,height=.75\linewidth]{./imgs/pie_t20_conv3_100043mat/l09-recon.png}&
\includegraphics[width=\linewidth,height=.75\linewidth]{./imgs/100043/l09-recon.png} &
\includegraphics[width=\linewidth,height=.75\linewidth]{./imgs/train_730_t20_conv3_100043mat/l09-recon.png}&
\includegraphics[width=\linewidth,height=.75\linewidth]{./imgs/train_730_t10_pool5_100043mat/l09-recon.png}&
\includegraphics[width=\linewidth,height=.75\linewidth]{./imgs/train_730_t10_fc7_100043mat/l09-recon.png}
& Inv($C3$)
\\
\includegraphics[width=\linewidth,height=.75\linewidth]{./imgs/pie/l15-recon}
&
\includegraphics[width=\linewidth,height=.75\linewidth]{./imgs/pie_t10_fc7_100043mat/l15-recon.png}&
\includegraphics[width=\linewidth,height=.75\linewidth]{./imgs/pie_t10_pool5_100043mat/l15-recon.png}&
\includegraphics[width=\linewidth,height=.75\linewidth]{./imgs/pie_t20_conv3_100043mat/l15-recon.png}&
\includegraphics[width=\linewidth,height=.75\linewidth]{./imgs/100043/l15-recon} &
\includegraphics[width=\linewidth,height=.75\linewidth]{./imgs/train_730_t20_conv3_100043mat/l15-recon.png}&
\includegraphics[width=\linewidth,height=.75\linewidth]{./imgs/train_730_t10_pool5_100043mat/l15-recon.png}&
\includegraphics[width=\linewidth,height=.75\linewidth]{./imgs/train_730_t10_fc7_100043mat/l15-recon.png}
& Inv($\text{P}5$)
\\
\includegraphics[width=\linewidth,height=.75\linewidth]{./imgs/pie/l16-recon.png}
&
\includegraphics[width=\linewidth,height=.75\linewidth]{./imgs/pie_t10_fc7_100043mat/l16-recon.png}&
\includegraphics[width=\linewidth,height=.75\linewidth]{./imgs/pie_t10_pool5_100043mat/l16-recon.png}&
\includegraphics[width=\linewidth,height=.75\linewidth]{./imgs/pie_t20_conv3_100043mat/l16-recon.png}&
\includegraphics[width=\linewidth,height=.75\linewidth]{./imgs/100043/l16-recon.png} &
\includegraphics[width=\linewidth,height=.75\linewidth]{./imgs/train_730_t20_conv3_100043mat/l16-recon.png}&
\includegraphics[width=\linewidth,height=.75\linewidth]{./imgs/train_730_t10_pool5_100043mat/l16-recon.png}&
\includegraphics[width=\linewidth,height=.75\linewidth]{./imgs/train_730_t10_fc7_100043mat/l16-recon.png}
& Inv($FC6$)
\\
\includegraphics[width=\linewidth,height=.75\linewidth]{./imgs/pie/l18-recon.png}
&
\includegraphics[width=\linewidth,height=.75\linewidth]{./imgs/pie_t10_fc7_100043mat/l18-recon.png}&
\includegraphics[width=\linewidth,height=.75\linewidth]{./imgs/pie_t10_pool5_100043mat/l18-recon.png}&
\includegraphics[width=\linewidth,height=.75\linewidth]{./imgs/pie_t20_conv3_100043mat/l18-recon.png}&
\includegraphics[width=\linewidth,height=.75\linewidth]{./imgs/100043/l18-recon.png} &
\includegraphics[width=\linewidth,height=.75\linewidth]{./imgs/train_730_t20_conv3_100043mat/l18-recon.png}&
\includegraphics[width=\linewidth,height=.75\linewidth]{./imgs/train_730_t10_pool5_100043mat/l18-recon.png}&
\includegraphics[width=\linewidth,height=.75\linewidth]{./imgs/train_730_t10_fc7_100043mat/l18-recon.png}
& Inv($\text{FC}7$)
\\
\hline
\end{tabular}
\caption{Sample inversions from Layer $l$ representation (Inv($l$)) of the adversarials optimized for different layers of Caffenet. Column $g$ is the inversion result of the guides and columns $s_1$ and $s_2$ are inversions of the source images. Input row is the image given as the input to the inverting convnet algorithm~\citep{MahendranVedaldiCVPR2015}. }
\label{fig:adv_invert}
\end{figure*}
\fi
}}}%%%%%%%%%%%%%%%%%%%%%%%%%%%%%%%%%%%%%%%%%%%%%%%%%%%%%%%%%%%%%%%%%%%%%%%%%%%%%%%%%%%%%%%%%%%%%%%%%%%%%%%%%%%%%%%%%%%%%%%%%%%%%%%%%%%%%%%%%%%%%%%%
\section{Experimental Evaluation}\label{sec:experiments}\vspace*{-0.2cm}% \subsubsection{Euclidean distance and nearest neighbors}% \label{NNAnalysis}% \subsection{Manifold local tangent space}% \label{lpca}We investigate further properties of adversarial images by asking two
questions. To what extent do internal representations of adversarial
images resemble those of the respective guides, and are the
representations unnatural in any obvious way?
To answer these questions we focus mainly on Caffenet, with random
pairs of source-guide images drawn from the ImageNet ILSVRC datasets.
% Further experiments in  Sec.\ \ref{sec:closeness} report results on % other well-known convolutional neural networks (CNNs).\comment{{{
To answer these questions quantitatively, we use several
measures of similarity described below in order to show that the adversarial
images we obtain have tested properties of natural image representations.
They do not appear to be outliers from the training corpus in any significant
way, as judged by the internal representations.
}}}% The set of sources comprise 20 images drawn at random ILSVRC training, % testing and validation sets. For the guide set, for each of 1000 classes % in ILSVRC, we draw three images from training and validation sets whoses % labels are correctly predicted by Caffenet (for simplicity in quantifying % classification behaviour of adversarial images).  The experiments comprise % all possible source-guide combinations from these sets.  % We start by inverting the reprsentation of the adversarials, following% \cite{MahendranVedaldiCVPR2015}, to show that although the purturbation is% impreceptible, but the inverted representation of the CNN looks almost % like the guide image.  % To answer the question regarding how natural an image representation looks,% we model the local% neighborhoods by various methods and show that not only the representation of% the adversarial is very close to the guide, but also it is an inlier in that% local neighborhood.% In what follows we demontrate the generation of adversarial images% and quantify many of their interesting properties, some of which % are evident above.  \iffalse\begin{figure}[t!]
%@{\qquad}
\centering
\begin{subfigure}[t]{0.72\linewidth}{\renewcommand{\arraystretch}{0}
\setlength\tabcolsep{0pt}
\begin{tabular}{@{}p{0.25\linewidth}p{0.25\linewidth}p{0.25\linewidth}p{0.25\linewidth}@{}}
\begin{tikzonimage}[width=\linewidth, height=.75\linewidth]
    {./imgs/train_378.png}[image label]
\node{Train};
\end{tikzonimage}&
\begin{tikzonimage}[width=\linewidth, height=.75\linewidth]{./imgs/val_73.png}[image label]
\node{Val};
\end{tikzonimage}&
\begin{tikzonimage}[width=\linewidth,
    height=.75\linewidth]{./imgs/test_61314.png}[image label]
\node{Test};
\end{tikzonimage}&
\begin{tikzonimage}[width=\linewidth, height=.75\linewidth]{./imgs/audi.png}[image label]
\node{Wiki};
\end{tikzonimage} \\
%\vspace*{0.05cm}
\includegraphics[width=\linewidth,height=.75\linewidth]{./imgs/train_821.png} &
\includegraphics[width=\linewidth,height=.75\linewidth]{./imgs/val_53.png} &
\includegraphics[width=\linewidth,height=.75\linewidth]{./imgs/test_7231.png} &
\includegraphics[width=\linewidth,height=.75\linewidth]{./imgs/pie.png}
\end{tabular}
}
\caption{Source}
\end{subfigure}
\begin{subfigure}[t]{.18\linewidth}{\renewcommand{\arraystretch}{0}
\begin{tabular}{@{}p{\linewidth}@{}}
\includegraphics[width=\linewidth, height=.75\linewidth]{./imgs/1.png} \\
%\vspace*{0.05cm}
\includegraphics[width=\linewidth, height=.75\linewidth]{./imgs/24.png}
\end{tabular}
}
\caption{Guide}
\end{subfigure}
\caption{Sample images from the source and guide sets used in the
experiments.  The labels shown indicate the source from which
they were drawn.}
\label{data_figure}
\end{figure}\fi%%%%%%%%%%%%%%%%%%%%%%%%%%%%%%%%%\comment{{{\begin{table}[]
\centering
\resizebox{\linewidth}{!}{%
\begin{tabular}{|c|c|c|c|c|c|}
\hline
 Data Type & Imagenet Training & Imagenet Validation & Imagenet Test & Wikipedia & Total \\ \hline
Source\# & $5$ (R) & $5$ (R) + $3$ (M) & $5$ (R) & $3$ (M) & $20$ \\
Guide\# & $3300$ (PR) & $30$(PR) & $100$(R) & $0$ & $3430$ \\ \hline
\end{tabular}
}
\caption{Number of images from each data type present in each data set.
R, PR and M stands for random, partially random and manual selection strategy.}
\label{data_table}
\end{table}
}}}%%%%%%%%%%%%%%%%%%%%%%%%%%%%%%%%%%This yields 6660 pairs, in total, across the%two disjoint guide sets.% The majority of our experimental results use the BVLC Caffe Refernce % model (caffenet).  A list of the layers of caffenet along with some % characteristics is given in Table~\ref{tab:caffenet}.  \comment{
\begin{table}[t]
\centering
\begin{tabular}{|m{0.45\linewidth}|m{0.15\linewidth}m{0.15\linewidth}m{0.15\linewidth}|}
    \hline
    Layer Name   & Pool 5 & FC 6 & FC 7 \\ \hline
    Dimension    & 9216 & 4096 & 4096 \\ \hline
    Mean of $avg_i$ & 1568.683    & 309.848    & 89.859   \\ \hline
    Maximum of $avg_i$ & 1904.253    & 418.522    & 115.334    \\ \hline
    Minimum of $avg_i$ & 1011.284    & 227.786    & 68.805    \\ \hline
    Average over all classes' maximum pairwise distance  & 2271.464    & 565.371    & 157.112    \\ \hline
\end{tabular}
\caption{Caffenet layer properties.
\david{move contents of table into the text in Section 4.2}}
\label{tab:caffenet}
\end{table}
}%\david{move this.} \yanshuai{Rephrased it, so maybe it's ok to put here %now...}%\sara{moved $\alpha$ source... to 2nd par 4.1}\comment{
As all analyses henceforth focus on the representation space rather than pixel
space, for notational convenience, we use $\adv$, $\source$ and $\guide$ to
denote DNN representations of the source, guide and adversarial images whenever
there is no confusion about the layer of the representations.
%\sara{moved $a_{ij}$}
}\comment{
Where appropriate, we use $\adv_{ij}^k$ to denote the DNN representation at
layer $k$, from the adversarial image built from source $i$ and guide $j$.
}\vspace*{-0.1cm}\subsection{Similarity to the Guide representation}\vspace*{-0.2cm}%\label{sec:closeness}

We first report quantitative measures of proximity between the source,
guide, and adversarial image encodings at intermediate layers.
Surprisingly, despite the constraint that forces adversarial and source
images to remain perceptually indistinguishable, the intermediate
representations of the adversarial images are much closer to guides
than source images.
More interestingly, the adversarial representations are often nearest
neighbors of their respective guides. We find this is true for a
remarkably wide range of natural images.

% To this end we report similarities between source, guide and adversarial % representations, in terms of common, interpretable distances in the % feature space, on a large collection of source-guide pairs.

For optimizations at layer FC7, we test on a dataset comprising over
20,000 source-guide pairs, sampled from training, test and validation sets
of ILSVRC, plus some images from Wikipedia to increase diversity.
For layers with higher dimensionality (e.g., P5), for computational
expedience, we use a smaller set of 2,000 pairs.  Additional details about
how images are sampled can be found in the supplementary material.
To simplify the exposition in what follows, we use $\source$, $\guide$ and
$\adv$  to denote DNN representations of source, guide and adversarial images,
whenever there is no confusion about the layer of the representations.

\begin{figure*}[t!]
\centering
\comment{\begin{subfigure}[t]{.26\linewidth}
\includegraphics[width=\linewidth]{./imgs/guide_init_all_p5.png}
\caption{\footnotesize{$\left. d(\adv,\!\guide) \middle/ d(\source,\!\guide) \right.$, P5}}
\label{fig:Guide_Init_p5}
\end{subfigure}
\begin{subfigure}[t]{.24\linewidth}
    \includegraphics[width=\linewidth, height=.8\linewidth]
    {./imgs/delta_layers.png}
\caption{\footnotesize{$\left. d(\adv,\!\guide) \middle/ d(\source,\!\guide)
\right.$, $\delta$}}
\end{subfigure}}
\begin{subfigure}[t]{.24\linewidth}
\includegraphics[width=\linewidth,
height=.78\linewidth]{./imgs/guide_init_all.png}
\caption{\footnotesize{$\left. d(\adv,\!\guide) \middle/ d(\source,\!\guide) \right.$}}
\label{fig:Guide_Init}
\end{subfigure}
\begin{subfigure}[t]{.24\linewidth}
\includegraphics[width=\linewidth,
height=.78\linewidth]{./imgs/guide_nneigh_all.png}
\caption{\footnotesize{$\left. {d(\adv,\!\guide)} \middle/ {\,\overline{d_1}(\guide)} \right.$}}
\label{fig:guide_nneigh}
\end{subfigure}
\begin{subfigure}[t]{.24\linewidth}
\includegraphics[width=\linewidth,
height=.78\linewidth]{./imgs/seed_intra_all.png}
\caption{\footnotesize{
$\left. {d(\adv,\!\source)} \middle/ {\, \overline{d}(\source)} \right.$}}
\label{fig:Source_Intra}
\end{subfigure}
\vspace*{-0.15cm}
\caption{Histogram of the Euclidean distances between FC7 adversarial encodings
($\adv$) and corresponding source ($\source$) and  guide  ($\guide$),
for optimizations targetting FC7.
Here, $d(x,y)$ is the distance between $x$ and $y$, $\overline{d}(\source)$
denotes the average pairwise distances between points from images of the same
class as the source, and $\overline{d_1}(\guide)$ is the average distance
to the nearest neighbor encoding among images with the same class as the guide.
Histograms aggregate over all source-guide pairs.}
\label{fig:H_Dist}
\end{figure*}\vspace*{-0.2cm}\paragraph{Euclidean Distance:}
As a means of quantifying the qualitative results in Fig.\\ref{fig:adv_invert}, for a large ensemble of source-guide pairs,
all optimized at layer FC7, Fig.~\ref{fig:H_Dist}\subref{fig:Guide_Init}
shows a histogram of the ratio of Euclidean distance between adversarial
$\adv$ and guide $\guide$ in FC7, to the distance between source
$\source$ and guide $\guide$ in FC7.
Ratios less than 0.5 indicate that the adversarial FC7
encoding is closer to $\guide$ than $\source$.
While one might think that the $L_\infty$ norm constraint on the
perturbation will limit the extent to which adversarial encodings
can deviate from the source, we find that the optimization fails
to reduce the FC7 distance ratio to less than $0.8$ in only $0.1\%$
of pairs when $\delta=5$.
Figure \ref{fig:delta_layer} below shows that if we relax the
$L_\infty$ bound on the deviation from the source image, then $\alpha$
is even closer to $\guide$, and that adversarial encodings become
closer to $\guide$ as one goes from low to higher layers of a DNN.

% The distance between internal FC$7$ representations of source and guide pairs% is $110.24$, while the average FC$7$ distances between adversarial with% $\delta=10$ and guide images is just $13.82$.  between internal FC$7$% representations of sowurce and guide pairs is $110.24$, while the average% FC$7$ distances between adversarial with $\delta=10$ and guide images % Other histograms in Fig.~\ref{fig:H_Dist} show the proximity of adversarial% encodings to $\guide$ relative to distances between $\guide$ and its % FC7 nearest neighbors from among ILSVRC images.% % Further evaluations show that with $\delta=10$, on average the FC7 % distance between $\adv$ and $\guide$ is 78\% smaller than the average % FC7 distance between other images' representations and their NNs.

Figure~\ref{fig:H_Dist}\subref{fig:guide_nneigh} compares the FC$7$
distances between $\adv$ and $\guide$ to the average FC7 distance between
representations of all ILSVRC training images from the same class as
the guide and their FC7 nearest neighbors (NN).
Not only is $\adv$ often the 1-NN of $\guide$, but the distance between
$\adv$ and $\guide$ is much smaller than the distance between other
points and their NN in the same class.
Fig.~\ref{fig:H_Dist}\subref{fig:Source_Intra} shows that the FC7 distance
between $\adv$ and $\source$ is relatively large compared to typical
pairwise distances between FC7 encodings of images of the source class.
Only $8\%$ of adversarial images  (at $\delta=10$) are closer to their
source than the average pairwise FC7 distance within the source class.

% \yanshuai{The rest of this section is still very verbose.}% \subsubsection{Rank of Average Nearest Neighbors Measure}% \subsection{Distribution of Average Nearest Neighbor Ranks}\label{NNAnalysis}\vspace{-0.2cm}\paragraph{Intersection and Average Distance to Nearest Neighbors:}
Looking at one's nearest neighbors provides another measure of similarity.
It is useful when densities of points changes significantly through
feature space, in which case Euclidean distance may be less meaningful.
To this end we quantify similarity through rank statistics on near neighbors.
We take the average distance to a point's $K$ NNs as a scalar score
for the point.  We then rank that point along with all other points of
the same label class within the training set.  As such, the rank is a
non-parametric transformation of average distance, but independant of
the unit of distance.  We denote the rank of a point $x$ as $\rank{K}{x}$;
we use $K=3$ below.  Since $\adv$ is close to $\guide$ by construction,
we exclude $\guide$ when finding NNs for adversarial points $\adv$.

\comment{
We also consider rank statistics as a non-parametric way to quantify
similarity between internal representations of adversarial and guide images.
Along with the NN intersections, we compute the average distance to top $K$ NN.
We consider rank statistics as way of quantifying similarity between average
distances to the top NNs.  }

Table~\ref{tb:3nn} shows 3NN intersection as well as the difference in rank
between adversarial and guide encodings, $\rankdiff{3}(\adv,
\guide)=\rank{3}{\adv}-\rank{3}{\guide}$. When $\adv$ is close enough to
$\guide$, we expect the intersection to be high, and rank differences to be
small in magnitude. As shown in Table~\ref{tb:3nn}, in most cases they share
exactly the same 3NN; and in at least $50\%$ of cases their rank is more
similar than $90\%$ of data points in that class.  These results are for
sources and guides taken from the ILSVRC training set.  The same statistics
are observed for data from test or validation sets.

\comment{Beside networks trained on ImageNet and Places205 dataset, we also
test a network trained on Flickr Style dataset \citep{karayev2013recognizing}
which has $80,000$ images categorized into $20$ categories of photography
styles.  }\begin{table*}[t] \resizebox{\linewidth}{!}{\centering
        \begin{tabular}{|c|c|c|c|c|}
            %
            \hline Model & Layer & $\cap 3$NN $=3$ & $\cap 3$NN $\geq 2$
            &  $\rankdiff{3}$ median, [min, max] ($\%$)\\
            %
            \hline
            %
CaffeNet~\citep{jia2014caffe}&       FC$7$&  $71$&   $95$&   $-5.98, [-64.69, 0.00]$\\
AlexNet~\citep{krizhevsky2012imagenet}&        FC$7$&  $72$&   $97$&   $-5.64, [-38.39, 0.00]$\\
GoogleNet~\citep{szegedy2014going}&      pool5/$7\times7$\_s1&   $87$&   $100$&  $-1.94, [-12.87, 0.10]$\\
VGG CNN S~\citep{chatfield2014return}&      FC$7$&  $84$&   $100$&  $-3.34, [-26.34, 0.00]$\\
Places205 AlexNet~\citep{zhou2014learning}&      FC$7$&  $91$&   $100$&  $-1.24, [-18.20, 8.04]$\\
Places205 Hybrid~\citep{zhou2014learning}&       FC$7$&  $85$&   $100$&  $-1.25, [-8.96, 8.29]$\\
% Flickr Style~\citep{xiafine}&   FC$8$ Flickr&   $100$&  $100$&  $0.00,
            % [0.00,
            %0.08]$\\
% Flickr Style~\cite{xiafine}&   FC$7$&  $28$&   $56$&   $-21.44,        [-98.40, -0.11]$\\
\comment{%
\hline
CaffeNet&       FC$7$&  $70$&   $99$&   $-5.32, [-40.41, 0.00]$\\
AlexNet&        FC$7$&  $79$&   $99$&   $-5.92, [-42.98, 0.23]$\\
GoogleNet&      pool5/$7\times7$\_s1&   $97$&   $100$&  $-1.62, [-10.18, 0.51]$\\
VGG CNN S&      FC$7$&  $79$&   $99$&   $-3.44, [-59.66, 0.73]$\\
Places205 AlexNet&      FC$7$&  $91$&   $100$&  $-1.18, [-8.30, 3.25]$\\
Places205 Hybrid&       FC$7$&  $84$&   $98$&   $-1.38, [-7.39, 5.87]$\\
Flickr Style&   FC$8$ Flickr&   $100$&  $100$&  $0.00,  [0.00, 0.04]$\\
Flickr Style&   FC$7$&  $25$&   $55$&   $-29.16,        [-99.12, 8.68]$\\
\hline
CaffeNet&       FC$7$&  $78$&   $98$&   $-6.29, [-25.58, -0.08]$\\
}
\hline
%
\end{tabular}}
\vspace*{-0.1cm}
\caption{Results for comparison of nearest neighbors of the
adversarial and guide.  We randomly select $100$ pairs of guide and
source images such that the guide is classified correctly and the source is
classified to a different class. The optimization is done for a maximum of
$500$ iterations, with $\delta=10$. The statistics are in percentiles.
}
\label{tb:3nn}
\vspace*{-0.1cm}
%
\end{table*}\comment{ We have observed a peculiar case for Flickr FC$7$, where the
    optimization very quickly converges to local minima. At the same time, we
    are $100\%$ successful in making the representation exactly the same at
    FC$8$.  Interestingly, at the local minima we see that the
    $\rank{K}{\neigh{1}{\adv}}$ is almost zero which is indicative of a dense
    region where movement to any direction will make the representation farther
    from the guide. If we only look at the adversarials that have come close to
    the guide such that their $1$NN is the guide, the statistics for
$\rankdiff{3}$ is much better, $-10.28, [-37.35, -2.39]$.}\comment{%

This means that the classification boundaries learned by the network in this
representation is linear and in local neighborhoods, euclidean distance is
expected to be meaningful.

For each guide-source pair, we take the training data that is correctly
classified to the class of the guide and we call it the true-positives of the
guide class. For all the true-positives of the guide class and the adversarial,
we compute the average of their distance to their $3$-nearest-neighbors. We use
this to measure the structure of local neighborhoods around datapoints. For an
adversarial to be an inlier, it needs to be on the underlying manifold of the
guide class and therefore its average distance to its $3$-nearest-neighbors
should come from the same distribution. Also, if the adversarial is close
enough to the guide, it should have the same nearest-neighbors as the guide and
almost the same distance to them as the guide.

Figure~\ref{fig:rank_diff} shows the distribution of difference of ranks
between adversarial and the corresponding guide. This figure shows that the
distribution is concentrated around zero. As shown by the lines indicating the
median of the distribution, we see that for at least $50\%$ of the
adversarials, the difference in the rank with the guide is less than $5\%$ of
the ranks in the class of the guide. This shows that our adversarials, very
often enter a region where the density is very similar to the region of the
guide. This observation besides the closeness of the adversarial to the guide
shows that we are so close to the guide that to the nearest neighbors, there is
little difference between the guide and the adversarial.


\begin{figure}[h] \centering
    \includegraphics[width=.5\linewidth]{./imgs/rank_diff.png} \caption{For
        Caffenet, this shows the distribution of differences between the rank
        an adversarial and its guide, $\rankdiff{3}$. Different plots
        correspond to the split of the dataset that the source and guide are
    selected from.}\label{fig:rank_diff}
%
\end{figure}

}\subsection{Similarity to Natural representations}\label{sec:natural_reps}\vspace*{-0.2cm}

Having established that internal representations of adversarial
images ($\adv$) are close to those of guides ($\guide$), we then ask,
to what extent are they typical of natural images?
That is, in the vicinity of $\guide$, is $\adv$ an inlier,
with the same characteristics as other points in the neighborhood?
We answer this question by examining two neighborhood properties:
1) a probabilistic parametric measure giving the log likelihood of a
point relative to the local manifold at $\guide$; 2) a geometric
non-parametric measure inspired by high dimensional outlier detection methods.

For the analysis that follows, let $\Neighs{K}{x}$ denote the set of
$K$ NNs of point $x$.  Also, let $N_{ref}$ be a set of reference points
comprising $15$ random points from $\Neighs{20}{\guide}$, and let $N_{c}$
be the remaining ``close'' NNs of the guide, $N_{c} = \Neighs{20}{\guide}
\setminus N_{ref}$.  Finally, let $N_f = \Neighs{50}{\guide} \setminus
\Neighs{40}{\guide}$ be the set of ``far'' NNs of the guide.
The reference set $N_{ref}$ is used for measurement construction, while
$\adv$, $N_c$ and $N_f$ are scored relative to $\guide$
by the two measures mentioned above. Because we use up to $50$ NNs, for
which Euclidean distance might not be meaningful similarity measure for
points in a high-dimensional space like P5, we use cosine distance for
defining NNs.
(The source images used below are the same $20$ used in
Sec.\\ref{sec:closeness}. For expedience, the guide set is a smaller version
of that used in Sec.\\ref{sec:closeness}, comprising three images from each
of only $30$ random classes.)

% \subsubsection{Manifold tangent space}% \label{sec:tangent_space}\vspace*{-0.2cm}\paragraph{Manifold Tangent Space:}
We build a probabilistic subspace model with probabilistic PCA (PPCA)
around $\guide$ and compare the likelihood of $\adv$ to other points.
More precisely, PPCA is applied to $N_{ref}$, whose principal space is
a secant plane that has approximately the same normal direction as the
tangent plane, but generally does not pass through $\guide$ because of
the curvature of the manifold. We correct this small offset by shifting the
plane to pass through $\guide$; with PPCA this is achieved by moving
the mean of the high-dimensional Gaussian to $\guide$.  We then evaluate
the log likelihood of points under the model, relative to the log likelihood
of $\guide$, denoted $\dlike{\cdot}{\guide}=L({\cdot}) - L({\guide})$.
We repeat this measurement for a large number of guide and source pairs,
and compare the distribution of $\Delta L$ for $\adv$ with points in
$N_c$ and $N_f$.

For guide images sampled from ILSVRC training and validation sets,
results for FC7 and P5 are shown in the first two columns of Fig.\\ref{fig:local_inlier}. Since the Gaussian is centred at $\guide$, $\Delta L$
is bounded above by zero. The plots show that $\adv$ is well explained
locally by the manifold tangent plane. Comparing $\adv$ obtained when
$\guide$ is sampled from training or validation sets
(Fig.\\ref{sf:dl_fc7_train} vs \ref{sf:dl_fc7_val}, \ref{sf:dl_pool5_train}
vs \ref{sf:dl_pool5_val}), we observe patterns very similar to those in
plots of the log likelihood under the local subspace models.  This suggests
that the phenomenon of adversarial perturbation in Eqn.~(\ref{adv_objective})
is an intrinsic property of the representation itself,
rather than the generalization of the model.

% \subsubsection{Angular consistency measure}% \label{sec:angular_const}\vspace*{-0.2cm}\paragraph{Angular Consistency Measure:}
If the NNs of $\guide$ are sparse in the high-dimensional feature space, or
the manifold has high curvature, a linear Gaussian model will be a poor fit.
So we consider a way to test whether $\adv$ is an inlier in the
vicinity of $\guide$ that does not rely on a manifold assumption.
We take a set of reference points near a $\guide$, $N_{ref}$, and measure
directions from $\guide$ to each point.  We then compare the directions
from $\guide$ with those from $\adv$ and other nearby points,
e.g., in $N_{c}$ or $N_{f}$, to see whether $\adv$ is similar to
other points around $\guide$ in terms of {\em angular consistency}.
Compared to points within the local manifold, a point far from the
manifold will tend to exhibit a narrower range of directions to others
points in the manifold.
Specifically, given reference set $N_{ref}$, with cardinality $k$, and with $z$
being $\adv$ or a point from $N_c$ or $N_f$,
our angular consistency measure is defined as
\begin{equation}
\Omega(z,\guide) \, = \,
\frac{1}{k}
\sum_{ x_i \in N_{ref} }
\frac{\langle x_i - z ,\, x_i - \guide \rangle}
{ \| x_i - z \| \, \| x_i-\guide \| }
\label{eq:omega_def}
\end{equation}% Define $v_i(z)=x_i - z$ to be the vector from $z$ to $x_i$; and similarly % for $v_i(\guide)$. % Then the angular consistency $\Omega(z,\guide)$ % between $z$ and $\guide$ with respect to the reference set is defined as:% \begin{equation}% \Omega(z,\guide) = \frac{1}{k}\sum^k_i{\frac{\langle v_i(z), v_i(\guide) \rangle}{ \|v_i(z)\| \|v_i(\guide)\| }} \label{eq:omega_def}% \end{equation}
Fig.\\ref{sf:om_fc7_train} and \ref{sf:om_pool5_train} show histograms
of $\Omega(\adv,\guide)$ compared to
$\Omega(n_c,\guide)$ where $n_c \in N_c$ and
$\Omega(n_f,\guide)$ where $n_f \in N_f$.
Note that maximum angular consistency is $1$, in which case the point
behaves like $\guide$. Other than differences in scaling and upper
bound, the angular consistency plots \ref{sf:om_fc7_train} and
\ref{sf:om_pool5_train} are strikingly similar to those for the likelihood
comparisons in the first two columns of Fig.\\ref{fig:local_inlier},
supporting the conclusion that $\adv$ is an inlier with respect to
representations of natural images.

%% \yanshuai{Add description about this measure: here, basically we are using the 15 random points of $N_{20}$ as reference landmarks, and draw rays from other points to them, and look at how corresponding rays differ for $g$ and another point by measuring cosine distance. An angular consistency value of 1 means the two points are probably the same.}\begin{figure}[t]
\centering
\begin{subfigure}[t]{.26\linewidth}
\includegraphics[width=\linewidth]{./imgs/localpca_improved_compare_true_nn__cosine_20_90_fc7_ll_hists.png}
\caption{$\Delta L$, FC7, $\guide \in$ training} \label{sf:dl_fc7_train}
\end{subfigure}
\begin{subfigure}[t]{.26\linewidth}
\includegraphics[width=\linewidth]{./imgs/localpca_improved_compare_true_nn__cosine_20_90_fc7_validation_ll_hists.png}
\caption{$\Delta L$, FC7, $\guide \in$ validation} \label{sf:dl_fc7_val}
\end{subfigure}
\hspace*{0.5cm}
\begin{subfigure}[t]{.26\linewidth}
\includegraphics[width=\linewidth]{./imgs/angle_diff_var_improved_compare_true_nn__cosine_20_fc7_angle_diff.png}
\caption{$\Omega$, FC7, $\guide \in$ training} \label{sf:om_fc7_train}
\end{subfigure}

\begin{subfigure}[t]{.26\linewidth}
\includegraphics[width=\linewidth]{./imgs/localpca_improved_compare_true_nn__cosine_20_90_pool5_ll_hists.png}
\caption{$\Delta L$, P5, $\guide \in$ training} \label{sf:dl_pool5_train}
\end{subfigure}
\begin{subfigure}[t]{.26\linewidth}
\includegraphics[width=\linewidth]{./imgs/localpca_improved_compare_true_nn__cosine_20_90_pool5_validation_ll_hists.png}
\caption{$\Delta L$, P5, $\guide \in$ validation} \label{sf:dl_pool5_val}
\end{subfigure}
\hspace*{0.5cm}
\begin{subfigure}[t]{.26\linewidth}
\includegraphics[width=\linewidth]{./imgs/angle_diff_var_improved_compare_true_nn__cosine_20_pool5_angle_diff.png}
\caption{$\Omega$, P5, $\guide \in$ training} \label{sf:om_pool5_train}
\end{subfigure}

\vspace*{-0.1cm}
\caption{Manifold inlier analysis: the first two columns
(\ref{sf:dl_fc7_train},\ref{sf:dl_fc7_val},\ref{sf:dl_pool5_train},\ref{sf:dl_pool5_val})
for results of manifold tangent space analysis, showing distribution of
difference in log likelihood of a point and $\guide$, $\Delta
L(\cdot,\guide)=L(\cdot)-L(\guide)$; the last column
(\ref{sf:om_fc7_train}),(\ref{sf:om_pool5_train}) for angular consistency
analysis, showing distribution of angular consistency $\Omega(\cdot,g)$,
between a point and $\guide$. See Eqn.\ \ref{eq:omega_def} for definitions.}
\label{fig:local_inlier}
\vspace*{-0.2cm}
\end{figure}%% \comment{comparing the log likelihood ($L$) of various points under a generative model locally built around the guide image feature. (Adv, \textcolor{blue}{BLUE}) $L(a)-L(g)$, (NNc, \textcolor{green}{GREEN}) $L(n_c)-L(g)$, (NNf, \textcolor{red}{RED}) $L(n_f)-L(g)$; where $g$ is a guide image feature; $a$ is an adversarial image feature; $n_c$ a random top-$20$ near neighbor of $g$ that is not used in building the local PPCA; $n_f$ a near neighbor of $g$ in the rank $21$ to $30$.}}\subsection{Comparisons and analysis}\label{sec:comparison}\vspace*{-0.2cm}

We now compare our feature adversaries to images created to optimize
mis-classification \citep{SzegedyElatICLR2014}, in part to illustrate
qualitative differences.
We also investigate if the linearity hypothesis for mis-classification
adversaries of \cite{GoodfellowEtalICLR2015} is consistent with and
explains  with our class of adversarial examples.
We hereby refer to our results as {\em feature adversaries via
optimization (feature-opt)}. The adversarial images designed to trigger
mis-classification via optimization \citep{SzegedyElatICLR2014},
described briefly in Sec.~\ref{related}, are referred to as
{\em label adversaries via optimization (label-opt)}.

\vspace*{-0.25cm}\paragraph{Comparison to label-opt:}\setcounter{subfigure}{0}\begin{figure*}[t!]
%\centering
%\begin{subfigure}[t]{.5\linewidth}
\centering
\renewcommand{\arraystretch}{1}
\setlength\tabcolsep{5pt}
\setlength{\abovecaptionskip}{5pt}
%\vspace*{-2.3cm}
\begin{subfigure}[t]{.001\linewidth}
\end{subfigure}
\begin{tabular}{
>{\centering\arraybackslash}m{0.6\linewidth}
>{\centering\arraybackslash}m{0.38\linewidth}}
%\hspace*{-1cm}
%\vspace{-0.45cm}
\setlength\tabcolsep{2pt}
\begin{tabular}{
>{\centering\arraybackslash}m{0.45\linewidth}
>{\centering\arraybackslash}m{0.45\linewidth}}
\includegraphics[width=\linewidth]{./imgs/rank_all_caffe.png} &
\includegraphics[width=\linewidth]{./imgs/rank_all_label_adv.png}
%\end{tabular}
\end{tabular}
%\vspace*{-0.15cm}
\captionof{subfigure}{Rank of adversaries vs rank of $n_1(\alpha)$:
Average distance of $3$-NNs is used to rank all points in predicted
class (excl.\ guide). Adversaries with same horizontal coordinate
share the same guide.}
\label{fig:scatter_rank}
&
%\hspace*{0.2cm}
\vspace*{-0.75cm}
\includegraphics[width=.7\linewidth]{./imgs/label-opt_1nn_ll_hists.png}
%\begin{subfigure}[t]{.245\linewidth}
\vspace*{0.15cm}
\captionof{subfigure}{Manifold analysis for label-opt adversaries, at layer
FC7, with tangent plane through $n_1(\alpha)$.}
\label{fig:labeladv_PPCA_Dist}

%\end{subfigure}
%\caption{\small{label-opt, Gaussian at $n_1(\adv)$}}
%% \begin{subfigure}[t]{.325\linewidth}
%% \begin{center}
%% \includegraphics[width=0.95\linewidth]{./imgs/localpca_improved_compare_true_nn__cosine_20_90_fc7fgrad_feat_ll_hists.png}
%% \end{center}
%% \caption{\small{feat-fgrad, Gaussian at $\guide$}}
%% \end{subfigure}
%% \caption{Manifold analysis for label-opt adversaries, at layer
%% FC7, with tangent plane through $n_1(\alpha)$}% \yanshuai{show this for label-fgrad and feat-fgrad}}


%% \begin{subfigure}[t]{.245\linewidth}
%% \includegraphics[width=\linewidth]{./imgs/rank_all_fgrad_label.png}
%% \end{subfigure}
%% \begin{subfigure}[t]{.245\linewidth}
%% \includegraphics[width=\linewidth]{./imgs/rank_all_fgrad_fc7.png}
%% \end{subfigure}
\end{tabular}
\vspace*{-0.2cm}
% To take care of the mess that tabular has made!
\addtocounter{figure}{-1}
\vspace*{-0.25cm}
\caption{Label-opt and feature-opt PPCA and rank measure comparison plots.}
\label{fig:compare}
\end{figure*}%\yanshuai{adv on labels and show rep on the layer below is out-of-subspace;%show this using lpca and sparsity pattern;%}%\yanshuai{previous TBD Understanding adv and causes" section removed and merged here.}%\yanshuai{Introduce fast gradient method}

To demonstrate that label-opt differs qualitatively from feature-opt,
we report three empirical results.
First, we rank $\adv$, $\guide$, and other points assigned the same
class label as $\guide$, according to their average distance to three
nearest neighbours, as in Sec.~\ref{sec:closeness}.
Fig.~\ref{fig:scatter_rank} shows rank of $\adv$ versus rank of its
nearest neighbor-$n_1(\adv)$ for the two types of adversaries. Unlike
feature-opt, for label-opt, the rank of $\adv$ does not correlate well
with the rank of $n_1(\adv)$.  In other words, for feature-opt $\adv$ is
close to $n_1(\adv)$, while for label-opt it is not.

Second, we use the manifold PPCA approach in Sec.~\ref{sec:natural_reps}.
Comparing to peaked histogram of standardized likelihood of feature-opt
shown in Fig.~\ref{fig:local_inlier}, Fig.~\ref{fig:labeladv_PPCA_Dist}
shows that label-opt examples are not represented well by the Gaussian
around the first NN of $\adv$.

%% And for the case of feat-fgrad, neither the Gaussian around the first NN nor $\guide$ explain %% $\adv$ well. \yanshuai{complete Fig. \ref{fig:labeladv_PPCA_Dist} with the %% subfigures needed for label-fgrad and feat-fgrad.}%%And for the case of feat-fgrad, it is not well explained by the Gaussian around $\guide$.

Third, we analyze the sparsity  patterns on different DNN layers for
different adversarial construction methods. It is well known that DNNs
with ReLU activation units produce sparse activations
(\cite{AISTATS2011_GlorotBB11}).  Therefore, if the degree of sparsity
increases after the adversarial perturbation, the adversarial example
is using additional paths to manipulate the resulting represenation.
We also investigate how many activated units are shared between the source
and the adversary, by computing the intersection over union {\em $I/U$} of
active units. If the $I/U$ is high on all layers, then two represenations
share most active paths. On the other hand, if $I/U$ is low,
while the degree of sparsity remains the same, then the adversary must have
closed some activation paths and opened new ones. In Table~\ref{tb:sparsity},
$\Delta S$ is the difference between the proportion of non-zero activations on
selected layers between the source image represenation for the two types
of adversaries. One can see that for all except FC$7$ of label-opt, the
difference is significant. The column ``$I/U$ with $\source$'' also shows that
feature-opt uses very different activation paths from $\source$ when compared
to label-opt.

%% In the last column ``$I/U$ with $\guide$'', feature-opt shows much %% higher $I/U$ with the %% guide representation on the targeted layer FC$7$ comparing to %% the one obtained by feat-fgrad.\vspace*{-0.25cm}\paragraph{Testing The Linearity Hypothesis for feature-opt:}\cite{GoodfellowEtalICLR2015} suggests that the existence of label adversaries
is a consequence of networks being too linear.  If this linearity hypothesis
applies to our class of adversaries, it should be possible to linearize the
DNN around the source image, and then obtain similar adversaries via
optimization. Formally, let $J_s = J(\phi(I_s))$ be the Jacobian matrix
of the internal layer encoding with respect to source image input.
Then, the
linearity hypothesis implies $\phi(I) \approx \phi(I_s) + J_s^{\T}(I-I_s)$.
Hence, we optimize $\| \phi(I_s) + J_s^{\T}(I-I_s) - \phi(I_g)\|^2_2$
subject to the same infinity norm constraint in Eqn.\\ref{infnorm_bound}.
We refer to these adversaries as {\em feature-linear}.

As shown in Fig.~\ref{fig:delta_layer}, such adversaries do not get
particularly close to the guide.  They get no closer than 80\%, while
for {\em feature-opt} the distance is reduced to $50\%$ or less
for layers down to C2.  Note that unlike {\em feature-opt}, the objective
of {\em feature-linear} does not guarantee a reduction in distance
when the constraint on $\delta$ is relaxed.  These results suggest that
the linearity hypothesis may not explain the existence of {\em feature-opt}
adversaries.

% Since {\em feature-linear} adversaries do not get particularly close % to the guide, we report no further analyses of the local neighborhood is done.% Also,% since the fast gradient method in \cite{GoodfellowEtalICLR2015} for adversarial generation is not  % relevant to whether the linearity hypothesis explains the % feature adversaries we just briefly discuss it in the supplements.%\cite{GoodfellowEtalICLR2015} also proposed a method to construct label %adversaries efficiently by taking a small step %consistent with the gradient.  While this {\em fast gradient} method shines %light on the label-opt misclassifications, and is useful for adversarial %training, it is not relevant to whether the linearity hypothesis explains the %feature adversaries.  Therefore we omit comparison to fast gradient method %results here, and discuss it in the Supplementary Material instead.%\yanshuai{put the new result with linearized opt here, and say that the distance is so large, that no further detailed local analysis is necessary.}%% One implication of observations of sparsity patterns is that the %% linear perturbation explanation of label adversarial examples in %% \citep{GoodfellowEtalICLR2015} does not seem to apply to our class of %% represenation adversarial examples. \fartash{Apply comments from Ian Goodfellow %%     and either remove fast-gradient method or rewrite this section to use %% results on the linearized version to make claims. } Because %% Table~\ref{tb:sparsity} shows that fast gradient methods produce adversaries %% that share almost the same active paths as the source image, regardless whether %% applied on labels or representatation. The fast gradient method is merely %% making exisiting active units change values, hence looks a lot more different %% from the guide represenation comparing to our results.\comment{
%% \begin{figure}[t]
%% \centering
%% %\begin{subfigure}[t]{.325\linewidth}
%% \begin{center}
%% \includegraphics[width=0.325\linewidth]{./imgs/label-opt_1nn_ll_hists.png}
%% \end{center}
%% \caption{\small{label-opt, Gaussian at $n_1(\adv)$}}
%% %\end{subfigure}

%% %% \begin{subfigure}[t]{.325\linewidth}
%% %% \begin{center}
%% %% \includegraphics[width=0.95\linewidth]{./imgs/label-fgrad_1nn_ll_hists.png}
%% %% \end{center}
%% %% \caption{\small{label-fgrad, Gaussian at $n_1(\adv)$}}
%% %% \end{subfigure}
%% %% \begin{subfigure}[t]{.24\linewidth}
%% %% \includegraphics[width=\linewidth]{./imgs/localpca_improved_compare_true_nn__cosine_20_90_fc7fgrad_feat_ll_hists.png}
%% %% \caption{\small{fc7, feat-fgrad, Gaussian at $n_1(\adv)$  (place-holder)}}
%% %% \end{subfigure}
%% %% \begin{subfigure}[t]{.325\linewidth}
%% %% \begin{center}
%% %% \includegraphics[width=0.95\linewidth]{./imgs/localpca_improved_compare_true_nn__cosine_20_90_fc7fgrad_feat_ll_hists.png}
%% %% \end{center}
%% %% \caption{\small{feat-fgrad, Gaussian at $\guide$}}
%% %% \end{subfigure}
%% \caption{Manifold analysis for label-opt adversaries, at layer
%% FC7, with tangent plane through $n_1(\alpha)$}% \yanshuai{show this for label-fgrad and feat-fgrad}}
%% \label{fig:labeladv_PPCA_Dist}
%% \end{figure}
}\begin{table*}[t]
\renewcommand{\arraystretch}{1}
\setlength\tabcolsep{2pt}
\setlength{\abovecaptionskip}{25pt}
%\vspace*{-2.3cm}
%\begin{subfigure}[t]{.001\linewidth}
%\end{subfigure}
\begin{tabular}{
>{\centering\arraybackslash}m{0.48\linewidth}
>{\centering\arraybackslash}m{0.02\linewidth}
>{\centering\arraybackslash}m{0.48\linewidth}}
\setlength\tabcolsep{2pt}
%\begin{table*}[t]
%    \resizebox{\linewidth}{!}{
%\centering
\vspace*{-0.2cm}
\begin{footnotesize}
\begin{tabular}{l|cc|cc|}
\cline{2-5}
   & \multicolumn{2}{c|}{$\Delta S$}   & \multicolumn{2}{c|}{I/U with s}  \\ \cline{2-5}
                            & \multicolumn{1}{c|}{feature-opt} & \multicolumn{1}{c|}{label-opt}  & \multicolumn{1}{c|}{feature-opt} & \multicolumn{1}{c|}{label-opt} \\ \hline
\multicolumn{1}{|l|}{FC7}  & $7 \pm 7$ & $13 \pm 5$ & $\boldsymbol{12 \pm 4}$ & $39 \pm 9$  \\ \cline{1-1}
\multicolumn{1}{|l|}{C5} & $0 \pm 1$ & $0 \pm 0$ &  $33 \pm 2$ & $70 \pm 5$   \\ \cline{1-1}
\multicolumn{1}{|l|}{C3} & $2 \pm 1$ & $0 \pm 0$ &   $60 \pm 1$ & $85 \pm 3$  \\ \cline{1-1}
\multicolumn{1}{|l|}{C1} &  $0 \pm 0$ & $0 \pm 0$ &   $78 \pm 0$ & $94 \pm 1$   \\ \hline
\end{tabular}
\end{footnotesize}
\vspace*{-0.5cm}
\captionof{table}{Sparsity analysis: Sparsity is quantified
as a percentage of the size of each layer.}
\label{tb:sparsity}& &
\setlength{\abovecaptionskip}{3pt}
%}
%\caption{Sparsity pattern analysis: all numbers in percentage.}
%\label{tb:sparsity} &
%\end{table*} &
%\vspace*{-.6cm}

\vspace*{-.2cm}
\includegraphics[width=.625\linewidth]{./imgs/delta_layers.png}
\captionof{figure}{Distance ratio $\left. d(\adv,\!\guide) \middle/
d(\source,\!\guide)\right.$ vs $\delta$.
C$2$, C$3$, P$5$, F$7$ are for {\it feature-opt} adversaries;
$\ell$-f7 denotes FC7 distances for {\it feature-linear}.}
\label{fig:delta_layer}
\end{tabular}
\vspace*{-0.5cm}
\end{table*}%% \begin{table*}[t]%%     \resizebox{\linewidth}{!}{\centering%% \begin{tabular}{l|cccc|cccc|cc|}%% \cline{2-11}%%    & \multicolumn{4}{c|}{$\Delta S$}   & \multicolumn{4}{c|}{I/U with s}  & \multicolumn{2}{c|}{I/U with g}  \\ \cline{2-11} %%                             & \multicolumn{1}{c|}{feat-opt} & \multicolumn{1}{c|}{label-opt} & \multicolumn{1}{c|}{feat-fgrad} & label-fgrad & \multicolumn{1}{c|}{feat-opt} & \multicolumn{1}{c|}{label-opt} & \multicolumn{1}{c|}{feat-fgrad} & label-fgrad & \multicolumn{1}{c|}{feat-opt} & \multicolumn{1}{c|}{feat-fgrad} \\ \hline%% \multicolumn{1}{|l|}{fc7}  & $7 \pm 7$ & $13 \pm 5$ & $3 \pm 1$ & $3 \pm 1$ & $\boldsymbol{12 \pm 4}$ & $39 \pm 9$ & $76 \pm 1$ & $75 \pm 2$ &  $\boldsymbol{68 \pm 10}$ & $11 \pm 4$ \\ \cline{1-1}%% \multicolumn{1}{|l|}{conv5} & $0 \pm 1$ & $0 \pm 0$ & $0 \pm 0$ & $0 \pm 0$ &  $33 \pm 2$ & $70 \pm 5$ & $88 \pm 0$ & $88 \pm 0$ & $11 \pm 3$ & $7 \pm 2$  \\ \cline{1-1}%% \multicolumn{1}{|l|}{conv3} &  $2 \pm 1$ & $0 \pm 0$ & $0 \pm 0$ & $0 \pm 0$ &   $60 \pm 1$ & $85 \pm 3$ & $94 \pm 0$ & $94 \pm 0$ & $25 \pm 3$ & $23 \pm 2$  \\ \cline{1-1}%% \multicolumn{1}{|l|}{conv1} &  $0 \pm 0$ & $0 \pm 0$ & $0 \pm 0$ & $0 \pm 0$ &   $78 \pm 0$ & $94 \pm 1$ & $96 \pm 0$ & $96 \pm 0$ & $34 \pm 2$ & $34 \pm 2$ \\ \hline%% \end{tabular}%% }%% \caption{Sparsity pattern analysis: all numbers in percentage.}%% \label{tb:sparsity}%% \end{table*}%% \begin{table*}[ht]%%     \resizebox{\linewidth}{!}{\centering%% \begin{subtable}[t]{\linewidth}%% \begin{tabular}{l|cccc|cccc|}%% \cline{2-9}%%    & \multicolumn{4}{c|}{$\Delta S$}   & \multicolumn{4}{c|}{I/U with s}  & \multicolumn{2}{c|}{I/U with g}  \\ \cline{2-9} %%                             & \multicolumn{1}{c|}{feat-adv} & \multicolumn{1}{c|}{label-opt} & \multicolumn{1}{c|}{feat-fgrad} & label-fgrad & \multicolumn{1}{c|}{feat-adv} & \multicolumn{1}{c|}{label-opt} & \multicolumn{1}{c|}{feat-fgrad} & label-fgrad \\ \hline%% \multicolumn{1}{|l|}{fc7}  & $0.07 \pm 0.07$ & $0.13 \pm 0.05$ & $0.03 \pm 0.01$ & $0.03 \pm 0.01$ & $0.12 \pm 0.04$ & $0.39 \pm 0.09$ & $0.76 \pm 0.01$ & $0.75 \pm 0.02$ \\ \cline{1-1}%% \multicolumn{1}{|l|}{conv5} & $0.00 \pm 0.01$ & $0.00 \pm 0.00$ & $-0.00 \pm 0.00$ & $-0.00 \pm 0.00$ &  $0.33 \pm 0.02$ & $0.70 \pm 0.05$ & $0.88 \pm 0.00$ & $0.88 \pm 0.00$ \\ \cline{1-1}%% \multicolumn{1}{|l|}{conv3} &  $0.02 \pm 0.01$ & $0.00 \pm 0.00$ & $-0.00 \pm 0.00$ & $-0.00 \pm 0.00$ &   $0.60 \pm 0.01$ & $0.85 \pm 0.03$ & $0.94 \pm 0.00$ & $0.94 \pm 0.00$ \\ \cline{1-1}%% \multicolumn{1}{|l|}{conv1} &  $-0.00 \pm 0.00$ & $0.00 \pm 0.00$ & $-0.00 \pm 0.00$ & $-0.00 \pm 0.00$ &   $0.78 \pm 0.00$ & $0.94 \pm 0.01$ & $0.96 \pm 0.00$ & $0.96 \pm 0.00$ \\ \hline%% \end{tabular}%% \end{subtable}%% \begin{subtable}[t]{\linewidth}%% \begin{tabular}{l|cc|}%% \cline{2-2}%%    & \multicolumn{4}{c|}{$\Delta S$}   & \multicolumn{4}{c|}{I/U with s}  & \multicolumn{2}{c|}{I/U with g}  \\ \cline{2-11} %%                             & \multicolumn{1}{c|}{feat-adv} & \multicolumn{1}{c|}{label-opt} & \multicolumn{1}{c|}{feat-fgrad} & label-fgrad & \multicolumn{1}{c|}{feat-adv} & \multicolumn{1}{c|}{label-opt} & \multicolumn{1}{c|}{feat-fgrad} & label-fgrad & \multicolumn{1}{c|}{feat-adv} & \multicolumn{1}{c|}{feat-fgrad} \\ \hline%% \multicolumn{1}{|l|}{fc7}  & $0.07 \pm 0.07$ & $0.13 \pm 0.05$ & $0.03 \pm 0.01$ & $0.03 \pm 0.01$ & $0.12 \pm 0.04$ & $0.39 \pm 0.09$ & $0.76 \pm 0.01$ & $0.75 \pm 0.02$ &  $0.68 \pm 0.10$ & $0.11 \pm 0.04$ \\ \cline{1-1}%% \multicolumn{1}{|l|}{conv5} & $0.00 \pm 0.01$ & $0.00 \pm 0.00$ & $-0.00 \pm 0.00$ & $-0.00 \pm 0.00$ &  $0.33 \pm 0.02$ & $0.70 \pm 0.05$ & $0.88 \pm 0.00$ & $0.88 \pm 0.00$ & $0.11 \pm 0.03$ & $0.07 \pm 0.02$  \\ \cline{1-1}%% \multicolumn{1}{|l|}{conv3} &  $0.02 \pm 0.01$ & $0.00 \pm 0.00$ & $-0.00 \pm 0.00$ & $-0.00 \pm 0.00$ &   $0.60 \pm 0.01$ & $0.85 \pm 0.03$ & $0.94 \pm 0.00$ & $0.94 \pm 0.00$ & $0.25 \pm 0.03$ & $0.23 \pm 0.02$  \\ \cline{1-1}%% \multicolumn{1}{|l|}{conv1} &  $-0.00 \pm 0.00$ & $0.00 \pm 0.00$ & $-0.00 \pm 0.00$ & $-0.00 \pm 0.00$ &   $0.78 \pm 0.00$ & $0.94 \pm 0.01$ & $0.96 \pm 0.00$ & $0.96 \pm 0.00$ & $0.34 \pm 0.02$ & $0.34 \pm 0.02$ \\ \hline%% \end{tabular}%% \end{subtable}%%     }%% \caption{Sparsity pattern analysis}%% \label{tb:sparsity}%% \end{table*}\vspace*{-0.25cm}\paragraph{Networks with Random Weights:}
We further explored whether the existence of {\em feature-opt} adversaries
is due to the learning algorithm and the training set, or to the structure
of deep networks per se. For this purpose, we randomly initialized layers of
Caffenet with orthonormal weights.  We then optimized for adversarial images
as above, and looked at distance ratios (as in Fig.\\ref{fig:H_Dist}).
Interestingly, the distance ratios for FC$7$ and Norm$2$ are similar to
Fig.~\ref{fig:delta_layer} with at most $2\%$ deviation.  On C$2$, the results
are at most $10\%$ greater than those on C$2$ for the trained Caffenet.
We note that both Norm$2$ and C$2$ are overcomplete representations
of the input. The table of distance ratios can be found in the
Supplementary Material.
These results with random networks suggest that the existence of
{\em feature-opt} adversaries may be a property of the network architecture.

\comment{This observation shows that the model itself is vulnerable to {\em
feature-opt} adversaries, rather than just a specific trained network.}% , whereas % our main experiments with trained networks showed various properties % of these adversaries that have practical consequences for computer vision. %\input{exp_understanding}
\section{Discussion}\vspace*{-0.1cm}

We introduce a new method for generating adversarial images that appear
perceptually similar to a given source image, but whose deep representations
mimic the characteristics of natural guide images.
Indeed, the adversarial images have representations at intermediate
layers appear quite natural and very much like the guide images used
in their construction.
We demonstrate empirically that these imposters capture the generic
nature of their guides at different levels of deep representations.  This
includes their proximity to the guide, and their locations in high density
regions of the feature space.
We show further that such properties are not shared by other
categories of adversarial images.

\comment{{{
The approach for generating adversarial images with internal representations
that mimic those of other images has worked well with a remarkably broad
class of images, including images from training and test sets, and images
of all classes in the Imagenet and Places datasets.
% Understanding the root causes of this phenomena remains unanswered.
%since the network is not particularly deep, and the data are not typical of

%% One possible cause may be that the fine-tuned net only exploits a subspace
%% of the FC$7$ representation, so during fine-tuning there may be distortions
%% to features outside that subspace that provide marginal gains in classification.
%% As a consequence, Euclidean distance in FC$7$ will no longer be a useful loss
%% function. That is, the fine-tuned representations are no longer generic
%% representations for natural images since features are projected out
%% before the final softmax layer.
%% There is some evidence in favour of this view, as we find that average distance of three NNs of the $\adv$ is always one of the smallest comparing to other points in the class, meaning that optimization for $\adv$ is somehow stuck in a high density region in the space. This could be due to the distortion of the representation space by fine-tuning.
%% \david{I am not sure this discussion of Flicker Style is
%% clear enough to justify some much space in the paper?}

%These failures suggest that the adversarial phenomena reported here depend both
%on having deep networks and a broad class of natural image inputs. Although receptive field
%\fartash{This needs support from the new experiments we did based on Ian
%Goodfellow’s comments}
}}}
We also find that the linearity hypothesis \citep{GoodfellowEtalICLR2015}
does not provide an obvious explanation for these new adversarial phenomena.
It appears that the existence of these adversarial images is not predicated
on a network trained with natural images per se.  For example, results
on random networks indicate that the structure of the network itself may
be one significant factor.  Nevertheless, further experiments and analysis
are required to determine the true underlying reasons for this discrepancy
between human and DNN representations of images.

Another future direction concerns the exploration of failure cases we
observed in optimizing feature adversaries. As mentioned in supplementary
material, such cases involve images of hand-written digits, and networks
that are fine-tuned with images from a narrow domain (e.g., the Flicker
Style dataset). Such failures suggest that our adversarial phenomena may
be due to factors such as network  depth, receptive field size, or the
class of natural images used.  Since our aim here was to analyze the
representation of well-known networks, we leave the exploration of these
factors to future work.
Another interesting question concerns whether existing discriminative
models might be trained to detect feature adversaries.  Since training
such models requires a diverse and relatively large dataset of adversarial
images we also leave this to future work.


\begin{footnotesize}
\paragraph{ACKNOWLEDGMENTS}  Financial support for this research was
provided, in part, by MITACS, NSERC Canada, and the Canadian Institute
for Advanced Research (CIFAR). We would like to thank Foteini Agrafioti
for her support. We would also like to thank Ian Goodfellow, Xavier Boix,
as well as the anoynomous reviewers for helpful feedback.
\end{footnotesize}
\bibliography{adv}
\bibliographystyle{iclr2016_conference}
\vfill

\newpage
\beginsupplement\section*{Supplementary Material}\label{}\subsection{Illustration of the idea}
Fig.~\ref{fig:illustrate} illustrates the achieved goal in this paper. The
image of the fancy car on the left is a training example from the ILSVRC
dataset. On the right of it, there is an adversarial image that was generated
by guiding the source image by an image of Max (the dog). While the two fancy
car images are very close in image space, the activation pattern of the
adversarial car is almost identical to that of Max.  This shows that the
mapping from the image space to the representation space is such that for each
natural image, there exists a point in a small neighborhood in the image space
that is mapped by the network to a point in the representation space that is in
a small neighborhood of the representation of a very different natural image.

\begin{figure}[h]
\centering
\includegraphics[width=\linewidth]{./imgs/technical-illustration-v4.jpg}
\caption{Summary of the main idea behind the paper.} \label{fig:illustrate}
\vspace*{-0.2cm}
\end{figure}\subsection{Datasets for Empirical Analysis}

Unless stated otherwise,  we have used the following two sets of source and
guide images. The first set is used for experiments on layer FC$7$ and the
second set is used for computational expedience on other layers (e.g.  P$5$).
The source images are guided by all guide images to show that the convergence
does not depend on the class of images. To simplify the reporting of
classification behavior, we only used guides from training set whose labels are
correctly predicted by Caffenet.

In both sets we used $20$ source images, with five drawn at random from each of
the ILSVRC train, test and validation sets, and five more selected manually
from Wikipedia and the ILSVRC validation set to provide greater diversity.  The
guide set for the first set consisted of three images from each of $1000$
classes, drawn at random from ILSVRC training images, and another $30$ images
from each of the validation and test sets. For the second set, we drew guide
images from just $100$ classes.

\subsection{Examples of Adversaries}
Fig.~\ref{fig:adv_caffenet_page} shows a random sample of source and guide
pairs along with their FC$7$ or Pool$5$ adversarial images. In none of the images the guide is perceptable in the adversary, regardless of the
choice of source, guide or layer. The only parameter that affects the
visibility of the noise is $\delta$.

\begin{figure*}[h!]
\centering
\renewcommand{\arraystretch}{1}
\setlength\tabcolsep{2pt}
\begin{tabular}{|
>{\centering\arraybackslash}m{0.15\linewidth}
>{\centering\arraybackslash}m{0.15\linewidth}
>{\centering\arraybackslash}m{0.15\linewidth}
>{\centering\arraybackslash}m{0.005\linewidth} |
>{\centering\arraybackslash}m{0.005\linewidth}
>{\centering\arraybackslash}m{0.15\linewidth}
>{\centering\arraybackslash}m{0.15\linewidth}
>{\centering\arraybackslash}m{0.15\linewidth}|}
\hline{\footnotesize Source} &
{\footnotesize $I_{\adv}^{\text{FC}7}\!,\delta\!=\!5$}  &{\footnotesize Guide}
& &
& {\footnotesize Source} &
{\footnotesize $I_{\adv}^{P5}\!,\delta\!=\!10$}  & {\footnotesize Guide}
\\[1ex] \hline

\includegraphics[width=\linewidth, height=.75\linewidth]{./imgs/seed_0.png} &
\includegraphics[width=\linewidth, height=.75\linewidth]{./imgs/ad_0.png} &
\includegraphics[width=\linewidth, height=.75\linewidth]{./imgs/guide_0.png}  &
& &
\includegraphics[width=\linewidth, height=.75\linewidth]{./imgs/seed_1.png} &
\includegraphics[width=\linewidth, height=.75\linewidth]{./imgs/ad_1.png} &
\includegraphics[width=\linewidth, height=.75\linewidth]{./imgs/guide_1.png}  \\
\includegraphics[width=\linewidth, height=.75\linewidth]{./imgs/seed_3.png} &
\includegraphics[width=\linewidth, height=.75\linewidth]{./imgs/ad_3.png} &
\includegraphics[width=\linewidth, height=.75\linewidth]{./imgs/guide_3.png}  &
& &
\includegraphics[width=\linewidth, height=.75\linewidth]{./imgs/seed_5.png} &
\includegraphics[width=\linewidth, height=.75\linewidth]{./imgs/ad_5.png} &
\includegraphics[width=\linewidth, height=.75\linewidth]{./imgs/guide_5.png}  \\
\includegraphics[width=\linewidth, height=.75\linewidth]{./imgs/seed_10.png} &
\includegraphics[width=\linewidth, height=.75\linewidth]{./imgs/ad_10.png} &
\includegraphics[width=\linewidth, height=.75\linewidth]{./imgs/guide_10.png}  &
 & &
\includegraphics[width=\linewidth, height=.75\linewidth]{./imgs/seed_12.png} &
\includegraphics[width=\linewidth, height=.75\linewidth]{./imgs/ad_12.png} &
\includegraphics[width=\linewidth, height=.75\linewidth]{./imgs/guide_12.png}  \\
\includegraphics[width=\linewidth, height=.75\linewidth]{./imgs/seed_14.png} &
\includegraphics[width=\linewidth, height=.75\linewidth]{./imgs/ad_14.png} &
\includegraphics[width=\linewidth, height=.75\linewidth]{./imgs/guide_14.png}  &
& &
\includegraphics[width=\linewidth, height=.75\linewidth]{./imgs/seed_16.png} &
\includegraphics[width=\linewidth, height=.75\linewidth]{./imgs/ad_16.png} &
\includegraphics[width=\linewidth, height=.75\linewidth]{./imgs/guide_16.png}  \\
\includegraphics[width=\linewidth, height=.75\linewidth]{./imgs/seed_20.png} &
\includegraphics[width=\linewidth, height=.75\linewidth]{./imgs/ad_20.png} &
\includegraphics[width=\linewidth, height=.75\linewidth]{./imgs/guide_20.png}  &
& &
\includegraphics[width=\linewidth, height=.75\linewidth]{./imgs/seed_21.png} &
\includegraphics[width=\linewidth, height=.75\linewidth]{./imgs/ad_21.png} &
\includegraphics[width=\linewidth, height=.75\linewidth]{./imgs/guide_21.png}  \\
\includegraphics[width=\linewidth, height=.75\linewidth]{./imgs/seed_22.png} &
\includegraphics[width=\linewidth, height=.75\linewidth]{./imgs/ad_22.png} &
\includegraphics[width=\linewidth, height=.75\linewidth]{./imgs/guide_22.png}  &
& &
\includegraphics[width=\linewidth, height=.75\linewidth]{./imgs/seed_25.png} &
\includegraphics[width=\linewidth, height=.75\linewidth]{./imgs/ad_25.png} &
\includegraphics[width=\linewidth, height=.75\linewidth]{./imgs/guide_25.png}  \\
\includegraphics[width=\linewidth, height=.75\linewidth]{./imgs/seed_27.png} &
\includegraphics[width=\linewidth, height=.75\linewidth]{./imgs/ad_27.png} &
\includegraphics[width=\linewidth, height=.75\linewidth]{./imgs/guide_27.png}  &
& &
\includegraphics[width=\linewidth, height=.75\linewidth]{./imgs/seed_30.png} &
\includegraphics[width=\linewidth, height=.75\linewidth]{./imgs/ad_30.png} &
\includegraphics[width=\linewidth, height=.75\linewidth]{./imgs/guide_30.png}  \\
\includegraphics[width=\linewidth, height=.75\linewidth]{./imgs/seed_34.png} &
\includegraphics[width=\linewidth, height=.75\linewidth]{./imgs/ad_34.png} &
\includegraphics[width=\linewidth, height=.75\linewidth]{./imgs/guide_34.png}  &
& &
\includegraphics[width=\linewidth, height=.75\linewidth]{./imgs/seed_36.png} &
\includegraphics[width=\linewidth, height=.75\linewidth]{./imgs/ad_36.png} &
\includegraphics[width=\linewidth, height=.75\linewidth]{./imgs/guide_36.png}  \\
\includegraphics[width=\linewidth, height=.75\linewidth]{./imgs/seed_39.png} &
\includegraphics[width=\linewidth, height=.75\linewidth]{./imgs/ad_39.png} &
\includegraphics[width=\linewidth, height=.75\linewidth]{./imgs/guide_39.png}  &
& &
\includegraphics[width=\linewidth, height=.75\linewidth]{./imgs/seed_44.png} &
\includegraphics[width=\linewidth, height=.75\linewidth]{./imgs/ad_44.png} &
\includegraphics[width=\linewidth, height=.75\linewidth]{./imgs/guide_44.png}  \\
\includegraphics[width=\linewidth, height=.75\linewidth]{./imgs/seed_47.png} &
\includegraphics[width=\linewidth, height=.75\linewidth]{./imgs/ad_47.png} &
\includegraphics[width=\linewidth, height=.75\linewidth]{./imgs/guide_47.png}  &
& &
\includegraphics[width=\linewidth, height=.75\linewidth]{./imgs/seed_50.png} &
\includegraphics[width=\linewidth, height=.75\linewidth]{./imgs/ad_50.png} &
\includegraphics[width=\linewidth, height=.75\linewidth]{./imgs/guide_50.png}  \\
\includegraphics[width=\linewidth, height=.75\linewidth]{./imgs/seed_57.png} &
\includegraphics[width=\linewidth, height=.75\linewidth]{./imgs/ad_57.png} &
\includegraphics[width=\linewidth, height=.75\linewidth]{./imgs/guide_57.png}  &
& &
\includegraphics[width=\linewidth, height=.75\linewidth]{./imgs/seed_61.png} &
\includegraphics[width=\linewidth, height=.75\linewidth]{./imgs/ad_61.png} &
\includegraphics[width=\linewidth, height=.75\linewidth]{./imgs/guide_61.png}  \\
\includegraphics[width=\linewidth, height=.75\linewidth]{./imgs/seed_62.png} &
\includegraphics[width=\linewidth, height=.75\linewidth]{./imgs/ad_62.png} &
\includegraphics[width=\linewidth, height=.75\linewidth]{./imgs/guide_62.png}  &
& &
\includegraphics[width=\linewidth, height=.75\linewidth]{./imgs/seed_66.png} &
\includegraphics[width=\linewidth, height=.75\linewidth]{./imgs/ad_66.png} &
\includegraphics[width=\linewidth, height=.75\linewidth]{./imgs/guide_66.png}  \\
%\includegraphics[width=\linewidth, height=.75\linewidth]{./imgs/seed_70.png} &
%\includegraphics[width=\linewidth, height=.75\linewidth]{./imgs/ad_70.png} &
%\includegraphics[width=\linewidth, height=.75\linewidth]{./imgs/guide_70.png}  &
%\includegraphics[width=\linewidth, height=.75\linewidth]{./imgs/seed_76.png} &
%\includegraphics[width=\linewidth, height=.75\linewidth]{./imgs/ad_76.png} &
%\includegraphics[width=\linewidth, height=.75\linewidth]{./imgs/guide_76.png}  \\
%\includegraphics[width=\linewidth, height=.75\linewidth]{./imgs/seed_80.png} &
%\includegraphics[width=\linewidth, height=.75\linewidth]{./imgs/ad_80.png} &
%\includegraphics[width=\linewidth, height=.75\linewidth]{./imgs/guide_80.png}  &
%\includegraphics[width=\linewidth, height=.75\linewidth]{./imgs/seed_85.png} &
%\includegraphics[width=\linewidth, height=.75\linewidth]{./imgs/ad_85.png} &
%\includegraphics[width=\linewidth, height=.75\linewidth]{./imgs/guide_85.png}  \\
%\includegraphics[width=\linewidth, height=.75\linewidth]{./imgs/seed_89.png} &
%\includegraphics[width=\linewidth, height=.75\linewidth]{./imgs/ad_89.png} &
%\includegraphics[width=\linewidth, height=.75\linewidth]{./imgs/guide_89.png}  &
%\includegraphics[width=\linewidth, height=.75\linewidth]{./imgs/seed_91.png} &
%\includegraphics[width=\linewidth, height=.75\linewidth]{./imgs/ad_91.png} &
%\includegraphics[width=\linewidth, height=.75\linewidth]{./imgs/guide_91.png}  \\
%\includegraphics[width=\linewidth, height=.75\linewidth]{./imgs/seed_92.png} &
%\includegraphics[width=\linewidth, height=.75\linewidth]{./imgs/ad_92.png} &
%\includegraphics[width=\linewidth, height=.75\linewidth]{./imgs/guide_92.png}  &
%\includegraphics[width=\linewidth, height=.75\linewidth]{./imgs/seed_95.png} &
%\includegraphics[width=\linewidth, height=.75\linewidth]{./imgs/ad_95.png} &
%\includegraphics[width=\linewidth, height=.75\linewidth]{./imgs/guide_95.png}  \\
%\includegraphics[width=\linewidth, height=.75\linewidth]{./imgs/seed_98.png} &
%\includegraphics[width=\linewidth, height=.75\linewidth]{./imgs/ad_98.png} &
%\includegraphics[width=\linewidth, height=.75\linewidth]{./imgs/guide_98.png}  &\hline
\hline
\end{tabular}
\caption{Each row shows examples of adversarial images, optimized
using different layers of Caffenet (FC$7$, P$5$), and different
values of $\delta=(5, 10)$.  }
\label{fig:adv_caffenet_page}
\vspace*{-0.1cm}
\end{figure*}%\input{fig_act_1}\subsection{Dimensionality of Representations}
The main focus of this study is on the well-known Caffenet model. The layer
names of this model and their representation dimensionalities are provided in
Tab.~\ref{tab:caffenet}.

\begin{table}[h!]
\resizebox{\linewidth}{!}{
\centering
\begin{tabular}{|l|c|c|c|c|c|c|}
    \hline
    Layer Name &Input & Conv$2$ & Norm$2$ & Conv$3$ & Pool$5$  & FC$7$ \\ \hline
    Dimensions & $3\times 227\times 227$  & $256\times 27\times 27$ & $256\times 13\times 13$
    & $384\times 13\times 13$ &  $256\times 6\times 6$ & $4096$ \\ \hline
    Total & $154587$ & $186624$ & $43264$ & $64896$ & $9216$ & $4096$ \\ \hline
 \end{tabular}
}
\caption{Caffenet layer dimensions.}
\label{tab:caffenet}
\end{table}\subsection{Results for Networks with Random Weights}
As described in Sec.~\ref{sec:comparison}, we attempt at analyzing the
architecture of Caffenet independent of the training by initializing the model
with random weights and generating feature adversaries. Results in
Tab.~\ref{tab:distance} show that we can generate feature adversaries on random
networks as well.  We use the ratio of distances of the adversary to the guide
over the source to the guide for this analysis. In each cell, the mean and
standard deviation of this ratio is shown for each of the three random,
orthonormal random and trained Caffenet networks.  The weights of the random
network are drawn from the same distribution that Caffenet is initialized with.
Orthorgonal random weights are obtained using singular value decomposition of the regular random weights.

Results in Tab.~\ref{tab:distance} indicate that convergence on Norm$2$ and
Conv$2$ is almost similar while the dimensionality of Norm$2$ is quite smaller
than Conv$2$. On the other hand, Fig.~\ref{fig:delta_layer} shows that although
Norm$2$ has smaller dimensionality than Conv$3$,  the optimization converges to
a closer point on Conv$3$ rather than Conv$2$ and hence Norm$2$.  This means
that the relation between dimensionality and the achieved distance of the
adversary is not straightforward.


\begin{table*}[h!]
\resizebox{\linewidth}{!}{
\centering
\begin{tabular}{|c|c|c|c|c|c|}
\hline
Layer & $\delta=5$ & $\delta=10$ & $\delta=15$ & $\delta=20$ & $\delta=25$ \\
\hline
conv2 &
\begin{tabular}[x]{@{}c@{}} T:$0.79 \pm 0.04$ \\OR:$0.89 \pm 0.03$\\ R:$0.90 \pm 0.02$\end{tabular}
&
\begin{tabular}[x]{@{}c@{}} T:$0.66 \pm 0.06$ \\OR:$0.78 \pm 0.05$\\ R:$0.81 \pm 0.04$\end{tabular}
&
\begin{tabular}[x]{@{}c@{}} T:$0.57 \pm 0.06$ \\OR:$0.71 \pm 0.07$\\ R:$0.74 \pm 0.06$\end{tabular}
&
\begin{tabular}[x]{@{}c@{}} T:$0.50 \pm 0.07$ \\OR:$0.64 \pm 0.09$\\ R:$0.67 \pm 0.08$\end{tabular}
&
\begin{tabular}[x]{@{}c@{}} T:$0.45 \pm 0.07$ \\OR:$0.58 \pm 0.10$\\ R:$0.61 \pm 0.09$\end{tabular}
\\
\hline
norm2 &
\begin{tabular}[x]{@{}c@{}} T:$0.80 \pm 0.04$ \\OR:$0.82 \pm 0.05$\\ R:$0.85 \pm 0.03$\end{tabular}
&
\begin{tabular}[x]{@{}c@{}} T:$0.66 \pm 0.05$ \\OR:$0.69 \pm 0.08$\\ R:$0.73 \pm 0.06$\end{tabular}
&
\begin{tabular}[x]{@{}c@{}} T:$0.57 \pm 0.06$ \\OR:$0.59 \pm 0.10$\\ R:$0.63 \pm 0.08$\end{tabular}
&
\begin{tabular}[x]{@{}c@{}} T:$0.50 \pm 0.06$ \\OR:$0.51 \pm 0.11$\\ R:$0.55 \pm 0.09$\end{tabular}
&
\begin{tabular}[x]{@{}c@{}} T:$0.45 \pm 0.06$ \\OR:$0.44 \pm 0.11$\\ R:$0.48 \pm 0.10$\end{tabular}
\\
\hline
fc7 &
\begin{tabular}[x]{@{}c@{}} T:$0.32 \pm 0.10$ \\OR:$0.34 \pm 0.12$\\ R:$0.52 \pm 0.09$\end{tabular}
&
\begin{tabular}[x]{@{}c@{}} T:$0.12 \pm 0.06$ \\OR:$0.12 \pm 0.09$\\ R:$0.26 \pm 0.11$\end{tabular}
&
\begin{tabular}[x]{@{}c@{}} T:$0.07 \pm 0.04$ \\OR:$0.07 \pm 0.06$\\ R:$0.13 \pm 0.10$\end{tabular}
&
\begin{tabular}[x]{@{}c@{}} T:$0.06 \pm 0.03$ \\OR:$0.05 \pm 0.04$\\ R:$0.07 \pm 0.08$\end{tabular}
&
\begin{tabular}[x]{@{}c@{}} T:$0.05 \pm 0.02$ \\OR:$0.05 \pm 0.02$\\ R:$0.04 \pm 0.06$\end{tabular}
\\
\hline
\end{tabular}
}
\caption{Ratio of  $\left. d(\adv,\!\guide) \middle/
d(\source,\!\guide)\right.$ as $\delta$ changes from $5$ to $25$ on randomly weighted(R), orthogonal randomly weighted(OR) and trained(T) Caffenet optimized on layers Conv2, Norm2 and FC7.}
\label{tab:distance}
\end{table*}\subsection{Adversaries by Fast Gradient}
As we discussed in Sec.~\ref{sec:comparison}, \cite{GoodfellowEtalICLR2015}
also proposed a method to construct label adversaries efficiently by taking
a small step consistent with the gradient.  While this {\em fast gradient}
method shines light on the label adversary misclassifications, and is useful
for adversarial training, it is not relevant to whether the linearity
hypothesis explains the feature adversaries.  Therefore we omitted the
comparison in Sec.~\ref{sec:comparison} to fast gradient method, and continue
the discussion here.

%\vspace*{1.0in}
The fast gradient method constructs adversaries (\cite{GoodfellowEtalICLR2015})
by taking the perturbation defined by $\delta \text{sign}(\nabla_I loss(f(I),
\ell ))$, where $f$ is the classifier, and $\ell$ is an erroneous label for
input image $I$.  We refer to the resulting adversarial examples {\em
label-fgrad}. We can also apply the fast gradient method to an internal
representation, i.e.\taking the perturbation defined by $\delta
\text{sign}(\nabla_I \| \phi(I) - \phi(I_g) \|^2)$. We call this
type {\em feature adversaries via fast gradient (feat-fgrad)}.

The same experimental setup as in Sec.~\ref{sec:comparison} is used here.  In
Fig.~\ref{fig:fgrad_plots}, we show the nearest neighbor rank analysis and
manifold analysis as done in Sec.~\ref{sec:natural_reps} and
Sec.~\ref{sec:comparison}.  Moreover,
Figs.~\ref{fig:manifold_label}-\ref{fig:manifold_feat} in compare to
Figs.~\ref{sf:dl_fc7_train}-\ref{sf:dl_fc7_val} from {\em feature-opt} results
and Fig.~\ref{fig:labeladv_PPCA_Dist} from {\em label-opt} results indicates
that this adversaries are not represented as well as {\em feature-opt} by
a Gaussian around the NN of the adversary too. Also,
Figs.~\ref{fig:rank_label}-\ref{fig:rank_feat} in compare to
Fig.~\ref{fig:scatter_rank} show the obvious difference in adversarial
distribution for the same set of source and guide.

\begin{figure}[h!]
\centering
\begin{subfigure}[t]{.325\linewidth}
\begin{center}
\includegraphics[width=0.95\linewidth]{./imgs/label-fgrad_1nn_ll_hists.png}
\end{center}
\caption{\small{label-fgrad, Gaussian at $n_1(\adv)$}}
\label{fig:manifold_label}
\end{subfigure}
\begin{subfigure}[t]{.325\linewidth}
\begin{center}
\includegraphics[width=0.95\linewidth]{./imgs/localpca_improved_compare_true_nn__cosine_20_90_fc7fgrad_feat_ll_hists.png}
\end{center}
\caption{\small{feat-fgrad, Gaussian at $\guide$}}
\label{fig:manifold_feat}
\end{subfigure}

\begin{subfigure}[t]{.325\linewidth}
\includegraphics[width=\linewidth]{./imgs/rank_all_fgrad_label.png}
\caption{\small{label-fgrad}}
\label{fig:rank_label}
\end{subfigure}
\begin{subfigure}[t]{.325\linewidth}
\includegraphics[width=\linewidth]{./imgs/rank_all_fgrad_fc7.png}
\caption{\small{feat-fgrad}}
\label{fig:rank_feat}
\end{subfigure}

\caption{Local property analysis of label-fgrad and feat-fgrad on FC7: \ref{fig:manifold_label}-\ref{fig:manifold_feat} manifold analysis; \ref{fig:rank_label}-\ref{fig:rank_feat} neighborhood rank analysis.}
\label{fig:fgrad_plots}
\end{figure}%% \begin{figure}[]%% \centering%% \begin{subfigure}[t]{.245\linewidth}%% \includegraphics[width=\linewidth]{./imgs/rank_all_fgrad_label.png}%% \caption{\small{label-fgrad}}%% \end{subfigure}%% \begin{subfigure}[t]{.245\linewidth}%% \includegraphics[width=\linewidth]{./imgs/rank_all_fgrad_fc7.png}%% \caption{\small{feat-fgrad}}%% \end{subfigure}%% \caption{Rank analysis for label-fgrad and feat-fgrad.}%% \end{figure}\subsection{Failure Cases}
There are cases in which our optimization was not successful in generating good
adversaries.  We observed that for low resolution images or hand-drawn
characters, the method does not always work well.  It was successful on LeNet
with some images from MNIST or CIFAR10, but for other cases we found it
necessary to relax the magnitude bound on the perturbations to the point that
traces of guide images were perceptible. With Caffenet, pre-trained on ImageNet
and then fine-tuned on the Flickr Style dataset, we could readily generate
adversarial images using FC$8$ in the optimization (i.e., the unnormalized
class scores), however, with FC$7$ the optimization often terminated without
producing adversaries close to guide images.  One possible cause may be that
the fine-tuning distorts the original natural image representation to benefit
style classification. As a consequence, the FC$7$ layer no longer gives a good
generic image represenation, and Euclidean distance on FC$7$ is no longer
useful for the loss function.

% \cite{karayev2013recognizing}\subsection{More Examples with Activation Patterns}
Finally, we dedicate the remaining pages to several pairs of source and guide
along with their adversaries, activation patterns and inverted images as a
complementary to Fig.~\ref{fig:adv_invert}. Figs.~\ref{fig:adv_invert2},
\ref{fig:adv_invert4}, \ref{fig:adv_invert3}, \ref{fig:adv_invert5} and \ref{fig:adv_invert6}
all have similar setup as it is discussed in Sec.~\ref{method}.

\begin{figure}[h!]
    \centering
\begin{subfigure}[t]{\linewidth}{
\renewcommand{\arraystretch}{1}
\setlength\tabcolsep{2pt}

\begin{tabular}{|
>{\centering\arraybackslash}m{0.09\linewidth} |
>{\centering\arraybackslash}m{0.167\linewidth} |
>{\centering\arraybackslash}m{0.167\linewidth}
>{\centering\arraybackslash}m{0.167\linewidth}
>{\centering\arraybackslash}m{0.167\linewidth} |
>{\centering\arraybackslash}m{0.167\linewidth} | }
\hline
 & Source & $\text{FC}7$ & $\text{P}5$ & C$3$ &Guide  \\\hline  Input
 & \includegraphics[width=\linewidth,height=.75\linewidth]{./imgs/pie.png}
& \includegraphics[width=\linewidth,height=.75\linewidth]{./imgs/pie_t10_fc7_100043mat/orig.png}
&
\includegraphics[width=\linewidth,height=.75\linewidth]{./imgs/pie_t10_pool5_100043mat/orig.png} &
\includegraphics[width=\linewidth,height=.75\linewidth]{./imgs/pie_43_conv3_t15/l10-orig.png} &
\includegraphics[width=\linewidth,height=.75\linewidth]{./imgs/val_43.png} \\
Inv($C3$) &
\includegraphics[width=\linewidth,height=.75\linewidth]{./imgs/pie/l10-recon.png} &
\includegraphics[width=\linewidth,height=.75\linewidth]{./imgs/pie_t10_fc7_100043mat/l10-recon.png} &
\includegraphics[width=\linewidth,height=.75\linewidth]{./imgs/pie_t10_pool5_100043mat/l10-recon.png} &
\includegraphics[width=\linewidth,height=.75\linewidth]{./imgs/pie_43_conv3_t15/l10-recon.png} &
\includegraphics[width=\linewidth,height=.75\linewidth]{./imgs/100043/l10-recon.png}
\\
Inv($P5$) &
\includegraphics[width=\linewidth,height=.75\linewidth]{./imgs/pie/l15-recon.png} &
\includegraphics[width=\linewidth,height=.75\linewidth]{./imgs/pie_t10_fc7_100043mat/l15-recon.png} &
\includegraphics[width=\linewidth,height=.75\linewidth]{./imgs/pie_t10_pool5_100043mat/l15-recon.png} &
\includegraphics[width=\linewidth,height=.75\linewidth]{./imgs/pie_43_conv3_t15/l15-recon.png} &
\includegraphics[width=\linewidth,height=.75\linewidth]{./imgs/100043/l15-recon.png}
\\
Inv($FC7$) &
\includegraphics[width=\linewidth,height=.75\linewidth]{./imgs/pie/l19-recon.png} &
\includegraphics[width=\linewidth,height=.75\linewidth]{./imgs/pie_t10_fc7_100043mat/l19-recon.png} &
\includegraphics[width=\linewidth,height=.75\linewidth]{./imgs/pie_t10_pool5_100043mat/l19-recon.png} &
\includegraphics[width=\linewidth,height=.75\linewidth]{./imgs/pie_43_conv3_t15/l19-recon.png} &
\includegraphics[width=\linewidth,height=.75\linewidth]{./imgs/100043/l19-recon.png}
\\

\hline
\end{tabular}
}
\end{subfigure}
\vspace*{0.2cm}

\begin{subfigure}[t]{\linewidth}{
\centering
\renewcommand{\arraystretch}{1}
\setlength\tabcolsep{.1pt}
\begin{tabular}{
|>{\centering\arraybackslash}m{0.205\linewidth}
>{\centering\arraybackslash}m{0.205\linewidth}
>{\centering\arraybackslash}m{0.205\linewidth}|
>{\centering\arraybackslash}m{0.125\linewidth}
>{\centering\arraybackslash}m{0.125\linewidth}
>{\centering\arraybackslash}m{0.125\linewidth}|
}
\hline
\includegraphics[width=\linewidth]{./imgs/f7_pie.png} &
\includegraphics[width=\linewidth]{./imgs/f7_pie_43.png} &
\includegraphics[width=\linewidth]{./imgs/f7_43.png} &
\includegraphics[height=\linewidth, angle=90]{./imgs/p5_pie.png} &
\includegraphics[height=\linewidth, angle=90]{./imgs/p5_pie_43.png} &
\includegraphics[height=\linewidth, angle=90]{./imgs/p5_43.png}\\
Source & FC7 Advers. & Guide & Source & P5 Advers. & Guide \\ \hline
%\multicolumn{3}{|c|}{FC$7$ Activations} & \multicolumn{3}{c|}{ P$5$
%Activations  } \\
%\hline
\end{tabular}
}
\end{subfigure}
\caption{
    Inverted images and activation plot for a pair of source and guide image
    shown in the first row (Input). This figure has same setting as
    Fig.~\ref{fig:adv_invert}.
}

\label{fig:adv_invert2}
\end{figure}\begin{figure*}[h!]

\begin{subfigure}[t]{\linewidth}{
\renewcommand{\arraystretch}{1}
\setlength\tabcolsep{2pt}
\begin{tabular}{|
>{\centering\arraybackslash}m{0.09\linewidth} |
>{\centering\arraybackslash}m{0.167\linewidth} |
>{\centering\arraybackslash}m{0.167\linewidth}
>{\centering\arraybackslash}m{0.167\linewidth}
>{\centering\arraybackslash}m{0.167\linewidth} |
>{\centering\arraybackslash}m{0.167\linewidth} | }
\hline
& Source & $\text{FC}7$ & $\text{P}5$ & C$3$ &Guide  \\\hline
Input &
\includegraphics[width=\linewidth,height=.75\linewidth]{./imgs/pie.png} &
\includegraphics[width=\linewidth,height=.75\linewidth]{./imgs/pie_t10_fc7_27mat/orig.png} &
\includegraphics[width=\linewidth,height=.75\linewidth]{./imgs/pie_t10_pool5_27mat/orig.png} &
\includegraphics[width=\linewidth,height=.75\linewidth]{./imgs/pie_t20_conv3_27mat/orig.png} &
\includegraphics[width=\linewidth,height=.75\linewidth]{./imgs/27.png} \\
Inv(C$3$) &
\includegraphics[width=\linewidth,height=.75\linewidth]{./imgs/pie/l09-recon.png} &
\includegraphics[width=\linewidth,height=.75\linewidth]{./imgs/pie_t10_fc7_27mat/l09-recon.png} &
\includegraphics[width=\linewidth,height=.75\linewidth]{./imgs/pie_t10_pool5_27mat/l09-recon.png} &
\includegraphics[width=\linewidth,height=.75\linewidth]{./imgs/pie_t20_conv3_27mat/l09-recon.png} &
\includegraphics[width=\linewidth,height=.75\linewidth]{./imgs/27/l09-recon.png}
\\
Inv($\text{P}5$) &
\includegraphics[width=\linewidth,height=.75\linewidth]{./imgs/pie/l16-recon.png} &
\includegraphics[width=\linewidth,height=.75\linewidth]{./imgs/pie_t10_fc7_27mat/l16-recon.png} &
\includegraphics[width=\linewidth,height=.75\linewidth]{./imgs/pie_t10_pool5_27mat/l16-recon.png} &
\includegraphics[width=\linewidth,height=.75\linewidth]{./imgs/pie_t20_conv3_27mat/l16-recon.png} &
\includegraphics[width=\linewidth,height=.75\linewidth]{./imgs/27/l16-recon.png}
\\
Inv($\text{FC}7$) &
\includegraphics[width=\linewidth,height=.75\linewidth]{./imgs/pie/l18-recon.png} &
\includegraphics[width=\linewidth,height=.75\linewidth]{./imgs/pie_t10_fc7_27mat/l18-recon.png} &
\includegraphics[width=\linewidth,height=.75\linewidth]{./imgs/pie_t10_pool5_27mat/l18-recon.png} &
\includegraphics[width=\linewidth,height=.75\linewidth]{./imgs/pie_t20_conv3_27mat/l18-recon.png} &
\includegraphics[width=\linewidth,height=.75\linewidth]{./imgs/27/l18-recon.png}
\\
\hline
\end{tabular}
}
\end{subfigure}
\vspace*{0.2cm}

\begin{subfigure}[t]{\linewidth}{
\centering
\renewcommand{\arraystretch}{1}
\setlength\tabcolsep{.1pt}
\begin{tabular}{
|>{\centering\arraybackslash}m{0.205\linewidth}
>{\centering\arraybackslash}m{0.205\linewidth}
>{\centering\arraybackslash}m{0.205\linewidth}|
>{\centering\arraybackslash}m{0.125\linewidth}
>{\centering\arraybackslash}m{0.125\linewidth}
>{\centering\arraybackslash}m{0.125\linewidth}|
}
\hline
\includegraphics[width=\linewidth]{./imgs/f7_pie.png} &
\includegraphics[width=\linewidth]{./imgs/f7_pie_27.png} &
\includegraphics[width=\linewidth]{./imgs/f7_27.png} &
\includegraphics[height=\linewidth, angle=90]{./imgs/p5_pie.png} &
\includegraphics[height=\linewidth, angle=90]{./imgs/p5_pie_27.png} &
\includegraphics[height=\linewidth, angle=90]{./imgs/p5_27.png}\\
Source & FC7 Advers. & Guide & Source & P5 Advers. & Guide \\ \hline
%\multicolumn{3}{|c|}{FC$7$ Activations} & \multicolumn{3}{c|}{ P$5$
%Activations  } \\
%\hline
\end{tabular}
}
\end{subfigure}
\caption{
    Inverted images and activation plot for a pair of source and guide image
    shown in the first row (Input). This figure has same setting as
    Fig.~\ref{fig:adv_invert}.
}
\label{fig:adv_invert4}
\end{figure*}\begin{figure}[h!]
    \centering
\begin{subfigure}[t]{\linewidth}{
\renewcommand{\arraystretch}{1}
\setlength\tabcolsep{2pt}
\begin{tabular}{|
>{\centering\arraybackslash}m{0.09\linewidth} |
>{\centering\arraybackslash}m{0.167\linewidth} |
>{\centering\arraybackslash}m{0.167\linewidth}
>{\centering\arraybackslash}m{0.167\linewidth}
>{\centering\arraybackslash}m{0.167\linewidth} |
>{\centering\arraybackslash}m{0.167\linewidth} | }
\hline
& Source & $\text{FC}7$ & $\text{P}5$ & C$3$ &Guide  \\\hline Input
& \includegraphics[width=\linewidth,height=.75\linewidth]{./imgs/train_730.png}
& \includegraphics[width=\linewidth,height=.75\linewidth]{./imgs/train_730_t10_fc7_27mat/orig.png}
&
\includegraphics[width=\linewidth,height=.75\linewidth]{./imgs/train_730_t10_pool5_27mat/orig.png} &
\includegraphics[width=\linewidth,height=.75\linewidth]{./imgs/730_27_conv3_t15/l10-orig.png} &
\includegraphics[width=\linewidth,height=.75\linewidth]{./imgs/27.png} \\
Inv($C3$)
& \includegraphics[width=\linewidth,height=.75\linewidth]{./imgs/730/l10-recon.png}
& \includegraphics[width=\linewidth,height=.75\linewidth]{./imgs/train_730_t10_fc7_27mat/l10-recon.png}
&
\includegraphics[width=\linewidth,height=.75\linewidth]{./imgs/train_730_t10_pool5_27mat/l10-recon.png} &
\includegraphics[width=\linewidth,height=.75\linewidth]{./imgs/730_27_conv3_t15/l10-recon.png} &
\includegraphics[width=\linewidth,height=.75\linewidth]{./imgs/27/l10-recon.png}
\\
Inv($P5$)
& \includegraphics[width=\linewidth,height=.75\linewidth]{./imgs/730/l15-recon.png}
& \includegraphics[width=\linewidth,height=.75\linewidth]{./imgs/train_730_t10_fc7_27mat/l15-recon.png}
&
\includegraphics[width=\linewidth,height=.75\linewidth]{./imgs/train_730_t10_pool5_27mat/l15-recon.png} &
\includegraphics[width=\linewidth,height=.75\linewidth]{./imgs/730_27_conv3_t15/l15-recon.png} &
\includegraphics[width=\linewidth,height=.75\linewidth]{./imgs/27/l15-recon.png}
\\
Inv($FC7$)
& \includegraphics[width=\linewidth,height=.75\linewidth]{./imgs/730/l19-recon.png}
& \includegraphics[width=\linewidth,height=.75\linewidth]{./imgs/train_730_t10_fc7_27mat/l19-recon.png}
&
\includegraphics[width=\linewidth,height=.75\linewidth]{./imgs/train_730_t10_pool5_27mat/l19-recon.png} &
\includegraphics[width=\linewidth,height=.75\linewidth]{./imgs/730_27_conv3_t15/l19-recon.png} &
\includegraphics[width=\linewidth,height=.75\linewidth]{./imgs/27/l19-recon.png}
\\
\hline
\end{tabular}
}
\end{subfigure}

\vspace*{0.2cm}

\begin{subfigure}[t]{\linewidth}{
\centering
\renewcommand{\arraystretch}{1}
\setlength\tabcolsep{.1pt}
\begin{tabular}{
|>{\centering\arraybackslash}m{0.205\linewidth}
>{\centering\arraybackslash}m{0.205\linewidth}
>{\centering\arraybackslash}m{0.205\linewidth}|
>{\centering\arraybackslash}m{0.125\linewidth}
>{\centering\arraybackslash}m{0.125\linewidth}
>{\centering\arraybackslash}m{0.125\linewidth}|
}
\hline
\includegraphics[width=\linewidth]{./imgs/f7_730.png} &
\includegraphics[width=\linewidth]{./imgs/f7_730_27.png} &
\includegraphics[width=\linewidth]{./imgs/f7_27.png} &
\includegraphics[height=\linewidth, angle=90]{./imgs/p5_730.png} &
\includegraphics[height=\linewidth, angle=90]{./imgs/p5_730_27.png} &
\includegraphics[height=\linewidth, angle=90]{./imgs/p5_27.png}\\
Source & FC7 Advers. & Guide & Source & P5 Advers. & Guide \\ \hline
%\multicolumn{3}{|c|}{FC$7$ Activations} & \multicolumn{3}{c|}{ P$5$
%Activations  } \\
%\hline
\end{tabular}
}
\end{subfigure}
\caption{
    Inverted images and activation plot for a pair of source and guide image
    shown in the first row (Input). This figure has same setting as
    Fig.~\ref{fig:adv_invert}.
}
\label{fig:adv_invert3}
\end{figure}%\begin{subfigure}[t]{\linewidth}%{%\centering%\renewcommand{\arraystretch}{1}%\setlength\tabcolsep{.1pt}%\begin{tabular}{%|>{\centering\arraybackslash}m{0.205\linewidth}%>{\centering\arraybackslash}m{0.205\linewidth}%>{\centering\arraybackslash}m{0.205\linewidth}|%>{\centering\arraybackslash}m{0.125\linewidth}%>{\centering\arraybackslash}m{0.125\linewidth}%>{\centering\arraybackslash}m{0.125\linewidth}|%}%\hline%\includegraphics[width=\linewidth]{./imgs/f7_730.png} &%\includegraphics[width=\linewidth]{./imgs/f7_730_27.png} &%\includegraphics[width=\linewidth]{./imgs/f7_27.png} &%\includegraphics[height=\linewidth, angle=90]{./imgs/p5_730.png} &%\includegraphics[height=\linewidth, angle=90]{./imgs/p5_730_27.png} &%\includegraphics[height=\linewidth, angle=90]{./imgs/p5_27.png}\\%Source & FC7 Advers. & Guide & Source & P5 Advers. & Guide \\ \hline%%\multicolumn{3}{|c|}{FC$7$ Activations} & \multicolumn{3}{c|}{ P$5$ %%Activations  } \\%%\hline%\end{tabular}%}%\end{subfigure}%\caption{%    Inverted images and activation plot for a pair of source and guide image %    shown in the first row (Input). This figure has same setting as %    Fig.~\ref{fig:adv_invert}.%}%\label{fig:adv_invert5}%\end{figure}%%\begin{figure}[h]
    \centering
\begin{subfigure}[t]{\linewidth}{
\renewcommand{\arraystretch}{1}
\setlength\tabcolsep{2pt}

\begin{tabular}{|
>{\centering\arraybackslash}m{0.09\linewidth} |
>{\centering\arraybackslash}m{0.167\linewidth} |
>{\centering\arraybackslash}m{0.167\linewidth}
>{\centering\arraybackslash}m{0.167\linewidth}
>{\centering\arraybackslash}m{0.167\linewidth} |
>{\centering\arraybackslash}m{0.167\linewidth} | }
\hline
 & Source & $\text{FC}7$ & $\text{P}5$ & C$3$ &Guide  \\\hline
 Input
 & \includegraphics[width=\linewidth,height=.75\linewidth]{./imgs/banana.png}
 & \includegraphics[width=\linewidth,height=.75\linewidth]{./imgs/banana_car_fc7/l15-orig.png}
 &
\includegraphics[width=\linewidth,height=.75\linewidth]{./imgs/banana_car_pool5/l15-orig.png} &
\includegraphics[width=\linewidth,height=.75\linewidth]{./imgs/banana_car_conv3/l15-orig.png} &
\includegraphics[width=\linewidth,height=.75\linewidth]{./imgs/car.png} \\
Inv($C3$) &
\includegraphics[width=\linewidth,height=.75\linewidth]{./imgs/banana/l10-recon.png} &
\includegraphics[width=\linewidth,height=.75\linewidth]{./imgs/banana_car_fc7/l10-recon.png} &
\includegraphics[width=\linewidth,height=.75\linewidth]{./imgs/banana_car_pool5/l10-recon.png} &
\includegraphics[width=\linewidth,height=.75\linewidth]{./imgs/banana_car_conv3/l10-recon.png} &
\includegraphics[width=\linewidth,height=.75\linewidth]{./imgs/car/l10-recon.png}
\\
Inv($P5$) &
\includegraphics[width=\linewidth,height=.75\linewidth]{./imgs/banana/l15-recon.png} &
\includegraphics[width=\linewidth,height=.75\linewidth]{./imgs/banana_car_fc7/l15-recon.png} &
\includegraphics[width=\linewidth,height=.75\linewidth]{./imgs/banana_car_pool5/l15-recon.png} &
\includegraphics[width=\linewidth,height=.75\linewidth]{./imgs/banana_car_conv3/l15-recon.png} &
\includegraphics[width=\linewidth,height=.75\linewidth]{./imgs/car/l15-recon.png}
\\
Inv($FC7$) &
\includegraphics[width=\linewidth,height=.75\linewidth]{./imgs/banana/l19-recon.png} &
\includegraphics[width=\linewidth,height=.75\linewidth]{./imgs/banana_car_fc7/l19-recon.png} &
\includegraphics[width=\linewidth,height=.75\linewidth]{./imgs/banana_car_pool5/l19-recon.png} &
\includegraphics[width=\linewidth,height=.75\linewidth]{./imgs/banana_car_conv3/l19-recon.png} &
\includegraphics[width=\linewidth,height=.75\linewidth]{./imgs/car/l19-recon.png}
\\

 \hline
\end{tabular}
}
\end{subfigure}
\vspace*{0.2cm}

\begin{subfigure}[t]{\linewidth}{
\centering
\renewcommand{\arraystretch}{1}
\setlength\tabcolsep{.1pt}
\begin{tabular}{
|>{\centering\arraybackslash}m{0.205\linewidth}
>{\centering\arraybackslash}m{0.205\linewidth}
>{\centering\arraybackslash}m{0.205\linewidth}|
>{\centering\arraybackslash}m{0.125\linewidth}
>{\centering\arraybackslash}m{0.125\linewidth}
>{\centering\arraybackslash}m{0.125\linewidth}|
}
\hline
\includegraphics[width=\linewidth]{./imgs/f7_ban.png} &
\includegraphics[width=\linewidth]{./imgs/f7_ban_car.png} &
\includegraphics[width=\linewidth]{./imgs/f7_car.png} &
\includegraphics[height=\linewidth, angle=90]{./imgs/p5_ban.png} &
\includegraphics[height=\linewidth, angle=90]{./imgs/p5_ban_car.png} &
\includegraphics[height=\linewidth, angle=90]{./imgs/p5_car.png}\\
Source & FC7 Advers. & Guide & Source & P5 Advers. & Guide \\ \hline
%\multicolumn{3}{|c|}{FC$7$ Activations} & \multicolumn{3}{c|}{ P$5$
%Activations  } \\
%\hline
\end{tabular}
}
\end{subfigure}
\caption{
    Inverted images and activation plot for a pair of source and guide image
    shown in the first row (Input). This figure has same setting as
    Fig.~\ref{fig:adv_invert}.
}
\label{fig:adv_invert5}
\end{figure}\begin{figure}[h!]
    \centering

\begin{subfigure}[t]{\linewidth}{
\renewcommand{\arraystretch}{1}
\setlength\tabcolsep{2pt}
\begin{tabular}{|
>{\centering\arraybackslash}m{0.09\linewidth} |
>{\centering\arraybackslash}m{0.167\linewidth} |
>{\centering\arraybackslash}m{0.167\linewidth}
>{\centering\arraybackslash}m{0.167\linewidth}
>{\centering\arraybackslash}m{0.167\linewidth} |
>{\centering\arraybackslash}m{0.167\linewidth} | }
\hline
& Source & $\text{FC}7$ & $\text{P}5$ & C$3$ &Guide  \\\hline
Input
& \includegraphics[width=\linewidth,height=.75\linewidth]{./imgs/bear.png}
& \includegraphics[width=\linewidth,height=.75\linewidth]{./imgs/bear_car_fc7/l15-orig.png}
&
\includegraphics[width=\linewidth,height=.75\linewidth]{./imgs/bear_car_pool5/l15-orig.png} &
\includegraphics[width=\linewidth,height=.75\linewidth]{./imgs/bear_car_conv3/l15-orig.png} &
\includegraphics[width=\linewidth,height=.75\linewidth]{./imgs/car.png} \\
Inv($C3$) &
\includegraphics[width=\linewidth,height=.75\linewidth]{./imgs/bear/l10-recon.png} &
\includegraphics[width=\linewidth,height=.75\linewidth]{./imgs/bear_car_fc7/l10-recon.png} &
\includegraphics[width=\linewidth,height=.75\linewidth]{./imgs/bear_car_pool5/l10-recon.png} &
\includegraphics[width=\linewidth,height=.75\linewidth]{./imgs/bear_car_conv3/l10-recon.png} &
\includegraphics[width=\linewidth,height=.75\linewidth]{./imgs/car/l10-recon.png}
\\
Inv($P5$) &
\includegraphics[width=\linewidth,height=.75\linewidth]{./imgs/bear/l15-recon.png} &
\includegraphics[width=\linewidth,height=.75\linewidth]{./imgs/bear_car_fc7/l15-recon.png} &
\includegraphics[width=\linewidth,height=.75\linewidth]{./imgs/bear_car_pool5/l15-recon.png} &
\includegraphics[width=\linewidth,height=.75\linewidth]{./imgs/bear_car_conv3/l15-recon.png} &
\includegraphics[width=\linewidth,height=.75\linewidth]{./imgs/car/l15-recon.png}
\\
Inv($FC7$) &
\includegraphics[width=\linewidth,height=.75\linewidth]{./imgs/bear/l19-recon.png} &
\includegraphics[width=\linewidth,height=.75\linewidth]{./imgs/bear_car_fc7/l19-recon.png} &
\includegraphics[width=\linewidth,height=.75\linewidth]{./imgs/bear_car_pool5/l19-recon.png} &
\includegraphics[width=\linewidth,height=.75\linewidth]{./imgs/bear_car_conv3/l19-recon.png} &
\includegraphics[width=\linewidth,height=.75\linewidth]{./imgs/car/l19-recon.png}
\\

\hline
\end{tabular}
}
\end{subfigure}
\vspace*{0.2cm}

\begin{subfigure}[t]{\linewidth}{
\centering
\renewcommand{\arraystretch}{1}
\setlength\tabcolsep{.1pt}
\begin{tabular}{
|>{\centering\arraybackslash}m{0.205\linewidth}
>{\centering\arraybackslash}m{0.205\linewidth}
>{\centering\arraybackslash}m{0.205\linewidth}|
>{\centering\arraybackslash}m{0.125\linewidth}
>{\centering\arraybackslash}m{0.125\linewidth}
>{\centering\arraybackslash}m{0.125\linewidth}|
}
\hline
\includegraphics[width=\linewidth]{./imgs/f7_bear.png} &
\includegraphics[width=\linewidth]{./imgs/f7_bear_car.png} &
\includegraphics[width=\linewidth]{./imgs/f7_car.png} &
\includegraphics[height=\linewidth, angle=90]{./imgs/p5_bear.png} &
\includegraphics[height=\linewidth, angle=90]{./imgs/p5_bear_car.png} &
\includegraphics[height=\linewidth, angle=90]{./imgs/p5_car.png}\\
Source & FC7 Advers. & Guide & Source & P5 Advers. & Guide \\ \hline
%\multicolumn{3}{|c|}{FC$7$ Activations} & \multicolumn{3}{c|}{ P$5$
%Activations  } \\
%\hline
\end{tabular}
}
\end{subfigure}
\caption{
    Inverted images and activation plot for a pair of source and guide image
    shown in the first row (Input). This figure has same setting as
    Fig.~\ref{fig:adv_invert}.
}
\label{fig:adv_invert6}
\end{figure}%\input{fig_activation_advs}
\end{document}
